\lecture{10}{2023-03-07}{}

\section{Graph Coloring}

\begin{example}[exam scheduling]
	\label{ex:exam scheduling}
	Define \(G = (V, E)\) where \(V\) are courses and \(E\) are conflicts if a student takes both courses. Here a graph coloring can help us answer how to schedule classes to limit conflicts.
\end{example} 

\begin{definition}[vertex coloring]
	\label{def:vertex coloring}
	Is a function of \(V \to \mathbb{N}\)  s.t. if \(\{u,v\} \in E\) then \(c(u) \neq c(v)\)  
\end{definition}

\begin{definition}[chromatic number]
	\label{def:chromatic number}
	Represented \(\chi (G)\), the chromatic number is the minimum number of colors needed to color a graph.
\end{definition}  

\begin{lemma}
	\(\chi (G) \leq \Delta G + 1\) where \(\Delta G\) is the max degree of a vertex in \(G\)  
\end{lemma} 

\begin{theorem}[Brooks]
	\label{thm:brooks}
	\(\chi (G) = \Delta (G) + 1 \implies\) \(G\) is an odd cycle or complete
\end{theorem} 

\begin{definition}[clique number]
	\label{def:clique number}
	Represented \(\omega (G)\), the clique number is the largest number s.t. \(K_n\) is subgraph of \(G\) (this implies \(\chi (G) \geq n\))
\end{definition} 

\subsection{Mycielski Construction}
\begin{example}[mycielski construction]
	\label{ex:mycielski construction}
% https://q.uiver.app/?q=WzAsNyxbMCwxLCJ2XzEiXSxbMSwxLCJ2XzIiXSxbNCwwLCJ2XzEiXSxbNCwyLCJ2XzIiXSxbNiwwLCJ1XzEiXSxbOCwxLCJ3Il0sWzYsMiwidV8yIl0sWzAsMSwiIiwwLHsic3R5bGUiOnsiaGVhZCI6eyJuYW1lIjoibm9uZSJ9fX1dLFsyLDMsIiIsMCx7InN0eWxlIjp7ImhlYWQiOnsibmFtZSI6Im5vbmUifX19XSxbNCw1LCIiLDAseyJzdHlsZSI6eyJoZWFkIjp7Im5hbWUiOiJub25lIn19fV0sWzUsNiwiIiwwLHsic3R5bGUiOnsiaGVhZCI6eyJuYW1lIjoibm9uZSJ9fX1dLFs0LDMsIiIsMix7InN0eWxlIjp7ImhlYWQiOnsibmFtZSI6Im5vbmUifX19XSxbMiw2LCIiLDIseyJzdHlsZSI6eyJoZWFkIjp7Im5hbWUiOiJub25lIn19fV0sWzEsOCwiTSIsMCx7InNob3J0ZW4iOnsic291cmNlIjoyMCwidGFyZ2V0IjoyMH19XV0=
\[\begin{tikzcd}
	&&&& {v_1} && {u_1} \\
	{v_1} & {v_2} &&&&&&& w \\
	&&&& {v_2} && {u_2}
	\arrow[no head, from=2-1, to=2-2]
	\arrow[""{name=0, anchor=center, inner sep=0}, no head, from=1-5, to=3-5]
	\arrow[no head, from=1-7, to=2-9]
	\arrow[no head, from=2-9, to=3-7]
	\arrow[no head, from=1-7, to=3-5]
	\arrow[no head, from=1-5, to=3-7]
	\arrow["M", shorten <=18pt, shorten >=18pt, Rightarrow, from=2-2, to=0]
\end{tikzcd}\]

With construction that if \(G = (V, E)\) where \(V = \{v_1 \ldots v_2\}\) then \(M(G)\) has vertices \(\{v_1 \ldots v_n, u_1 \ldots u_n, w\}\) and edges \(\{v_i,v_j\}, \{v_i, u_j\}, \{u_i, w\}\), for \(\{v_i, v_j\} \in E\) 
\end{example}

\begin{theorem}[Mycielski]
	\label{thm:mycielski}
	\begin{enumerate}
		\item[]
		\item \(\chi (G) = k \implies \chi (M(G)) = k + 1\)
		\item \(G\) doesn't contain \(K_3\) \(\implies\) \(M(G)\) doesn't contain \(K_3\)   
	\end{enumerate}
\end{theorem}
\begin{proof}
	\emph{First statement:}
	Given k-coloring of G, we can (\(k+1\)) color \(M(G)\), where you color \(u_i\) the same as \(v_1\) and \(w\) the \(k+1\) color.
	We also want to show that the graph has no smaller \(k\) coloring. Suppose \(M(G)\) has a k-coloring then \(U\) uses a \(k-1\) coloring and since the \(U\) coloring can be sent to the \(V\) coloring. Therefore, \(G\) has a \(k-1\) coloring - a contradiction.
\end{proof}

\begin{remark}
	Therefore, the clique number is not a very strong property
\end{remark}

\subsection{Coloring Extremal Problem}

\begin{note}
	Coloring graphs is an NP problem, so it might be better to try to solve extremal problems
\end{note} 

\begin{question}[coloring extremal problems]
	Among graphs with a \(\chi (G) = k\) what is the maximal/minimal number of edges?
\end{question}