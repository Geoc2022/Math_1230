\lecture{11}{2023-03-09}{}

\begin{definition}[turan graphs]
	\label{def:turan graphs}
	Graphs of the form \(T_{n,k} = M_{\underset{k - r}{\underbrace{q \ldots q}}, \underset{r}{\underbrace{q + 1 \ldots q + 1}}}\) are the solutions to the maximal coloring problem.
\end{definition}
\begin{proof}
	Pretty obvious with calculus based approach
	% Come back to with hw solution
\end{proof}

\begin{example}
	\(T_{6,3} = M_{3,2,1}\) 
	% https://q.uiver.app/?q=WzAsNixbMSwyLCJcXGJ1bGxldCJdLFsxLDAsIlxcYnVsbGV0Il0sWzIsMCwiXFxidWxsZXQiXSxbMCwwLCJcXGJ1bGxldCJdLFs0LDEsIlxcYnVsbGV0Il0sWzQsMiwiXFxidWxsZXQiXSxbMywwLCIiLDAseyJzdHlsZSI6eyJoZWFkIjp7Im5hbWUiOiJub25lIn19fV0sWzEsMCwiIiwyLHsic3R5bGUiOnsiaGVhZCI6eyJuYW1lIjoibm9uZSJ9fX1dLFsyLDAsIiIsMix7InN0eWxlIjp7ImhlYWQiOnsibmFtZSI6Im5vbmUifX19XSxbNCwwLCIiLDIseyJzdHlsZSI6eyJoZWFkIjp7Im5hbWUiOiJub25lIn19fV0sWzUsMCwiIiwyLHsic3R5bGUiOnsiaGVhZCI6eyJuYW1lIjoibm9uZSJ9fX1dLFszLDUsIiIsMix7InN0eWxlIjp7ImhlYWQiOnsibmFtZSI6Im5vbmUifX19XSxbMyw0LCIiLDIseyJzdHlsZSI6eyJoZWFkIjp7Im5hbWUiOiJub25lIn19fV0sWzEsNSwiIiwyLHsic3R5bGUiOnsiaGVhZCI6eyJuYW1lIjoibm9uZSJ9fX1dLFsxLDQsIiIsMix7InN0eWxlIjp7ImhlYWQiOnsibmFtZSI6Im5vbmUifX19XSxbMiw1LCIiLDIseyJzdHlsZSI6eyJoZWFkIjp7Im5hbWUiOiJub25lIn19fV0sWzIsNCwiIiwyLHsic3R5bGUiOnsiaGVhZCI6eyJuYW1lIjoibm9uZSJ9fX1dXQ==
\[\begin{tikzcd}
	\bullet & \bullet & \bullet \\
	&&&& \bullet \\
	& \bullet &&& \bullet
	\arrow[no head, from=1-1, to=3-2]
	\arrow[no head, from=1-2, to=3-2]
	\arrow[no head, from=1-3, to=3-2]
	\arrow[no head, from=2-5, to=3-2]
	\arrow[no head, from=3-5, to=3-2]
	\arrow[no head, from=1-1, to=3-5]
	\arrow[no head, from=1-1, to=2-5]
	\arrow[no head, from=1-2, to=3-5]
	\arrow[no head, from=1-2, to=2-5]
	\arrow[no head, from=1-3, to=3-5]
	\arrow[no head, from=1-3, to=2-5]
\end{tikzcd}\] 
\end{example}

\subsection{Chromatic Polynomial}
\begin{definition}[chromatic polynomial]
	\label{def:chromatic polynomial}
	\(\chi (G, t) =\) number of colorings of \(G\) using at most \(t\) colors

	\begin{enumerate}
		\item Degree = \(|V|\)
		\item \(a_{n-1}\) = \(|E|\)
		\item \(\chi (G) = \) smallest \(t\) s.t. \(\chi (G, t) \neq 0\) (\(0, \ldots \chi (G) -1\) are roots of \(\chi (G, t)\))
		\item \(t^d\) divides \(\chi (G, t) \implies\) G has \(\geq d\) components
		\item coefficients are log concave 
	\end{enumerate}
\end{definition}

\begin{example}[chromatic polynomial of a complete graph]
	\label{ex:chromatic polynomial of a complete graph}
	\(\chi (K_n, t) = t(t-1)(t-2)\ldots (t -(n+1)) = {t \choose n}\) 
\end{example}

\begin{example}[chromatic polynomial of a tree]
	\label{ex:chromatic polynomial of a tree}
	Given a tree \(T\), and a subtree \(S\) with one less vertex. Inductively, \(\chi (T, t) = (t-1)\chi (S, t)\).  And the coloring does not depend on what tree it is, only the number of vertices. \(\chi (T, t) = t \cdot (t-1)^{n-1}\) 
\end{example}

\begin{theorem}
	The chromatic polynomial is a polynomial 
\end{theorem}
\begin{proof}
	\(\chi (G, t) = \chi (G \setminus e , t) = \chi (G \cdot e, t)\)
	
	Where \(G \cdot e\) is the contraction formula 
\end{proof}

\begin{definition}[contraction formula]
	\label{def:contraction formula}
% https://q.uiver.app/?q=WzAsMTMsWzAsMCwiXFxidWxsZXQiXSxbMCwxLCJcXGJ1bGxldCJdLFsxLDEsIlxcYnVsbGV0Il0sWzEsMCwiXFxidWxsZXQiXSxbMiwwLCJcXGJ1bGxldCJdLFszLDAsIlxcYnVsbGV0Il0sWzMsMSwiXFxidWxsZXQiXSxbNCwwLCJcXGJ1bGxldCJdLFs1LDAsIlxcYnVsbGV0Il0sWzYsMCwiXFxidWxsZXQiXSxbNiwxLCJcXGJ1bGxldCJdLFs3LDAsIlxcYnVsbGV0Il0sWzgsMCwiXFxidWxsZXQiXSxbMCwxLCIiLDAseyJzdHlsZSI6eyJoZWFkIjp7Im5hbWUiOiJub25lIn19fV0sWzEsMiwiIiwwLHsic3R5bGUiOnsiaGVhZCI6eyJuYW1lIjoibm9uZSJ9fX1dLFsyLDMsIiIsMCx7InN0eWxlIjp7ImJvZHkiOnsibmFtZSI6ImRhc2hlZCJ9LCJoZWFkIjp7Im5hbWUiOiJub25lIn19fV0sWzMsMCwiIiwwLHsic3R5bGUiOnsiaGVhZCI6eyJuYW1lIjoibm9uZSJ9fX1dLFszLDQsIiIsMCx7InN0eWxlIjp7ImhlYWQiOnsibmFtZSI6Im5vbmUifX19XSxbNCwyLCIiLDAseyJzdHlsZSI6eyJoZWFkIjp7Im5hbWUiOiJub25lIn19fV0sWzUsNiwiIiwwLHsic3R5bGUiOnsiaGVhZCI6eyJuYW1lIjoibm9uZSJ9fX1dLFs2LDcsIiIsMCx7InN0eWxlIjp7ImhlYWQiOnsibmFtZSI6Im5vbmUifX19XSxbNSw3LCIiLDIseyJzdHlsZSI6eyJoZWFkIjp7Im5hbWUiOiJub25lIn19fV0sWzcsOCwiIiwyLHsiY3VydmUiOjEsInN0eWxlIjp7ImhlYWQiOnsibmFtZSI6Im5vbmUifX19XSxbNyw4LCIiLDIseyJjdXJ2ZSI6LTEsInN0eWxlIjp7ImhlYWQiOnsibmFtZSI6Im5vbmUifX19XSxbOSwxMCwiIiwyLHsic3R5bGUiOnsiaGVhZCI6eyJuYW1lIjoibm9uZSJ9fX1dLFsxMCwxMSwiIiwyLHsic3R5bGUiOnsiaGVhZCI6eyJuYW1lIjoibm9uZSJ9fX1dLFs5LDExLCIiLDAseyJzdHlsZSI6eyJoZWFkIjp7Im5hbWUiOiJub25lIn19fV0sWzExLDEyLCIiLDAseyJzdHlsZSI6eyJoZWFkIjp7Im5hbWUiOiJub25lIn19fV0sWzE4LDE5LCIiLDAseyJzaG9ydGVuIjp7InNvdXJjZSI6MjAsInRhcmdldCI6MjB9fV0sWzIwLDI0LCIiLDAseyJzaG9ydGVuIjp7InNvdXJjZSI6MjAsInRhcmdldCI6MjB9fV1d
\[\begin{tikzcd}
	\bullet & \bullet & \bullet & \bullet & \bullet & \bullet & \bullet & \bullet & \bullet \\
	\bullet & \bullet && \bullet &&& \bullet
	\arrow[no head, from=1-1, to=2-1]
	\arrow[no head, from=2-1, to=2-2]
	\arrow[dashed, no head, from=2-2, to=1-2]
	\arrow[no head, from=1-2, to=1-1]
	\arrow[no head, from=1-2, to=1-3]
	\arrow[""{name=0, anchor=center, inner sep=0}, no head, from=1-3, to=2-2]
	\arrow[""{name=1, anchor=center, inner sep=0}, no head, from=1-4, to=2-4]
	\arrow[""{name=2, anchor=center, inner sep=0}, no head, from=2-4, to=1-5]
	\arrow[no head, from=1-4, to=1-5]
	\arrow[curve={height=6pt}, no head, from=1-5, to=1-6]
	\arrow[curve={height=-6pt}, no head, from=1-5, to=1-6]
	\arrow[""{name=3, anchor=center, inner sep=0}, no head, from=1-7, to=2-7]
	\arrow[no head, from=2-7, to=1-8]
	\arrow[no head, from=1-7, to=1-8]
	\arrow[no head, from=1-8, to=1-9]
	\arrow[shorten <=10pt, shorten >=10pt, Rightarrow, from=0, to=1]
	\arrow[shorten <=16pt, shorten >=16pt, Rightarrow, from=2, to=3]
\end{tikzcd}\]
\end{definition}  

\begin{example}[sudoku]
	\label{ex:sudoku}
	We form a graph where vertices are squares on the board and vertices are connected if they are in the same row column or square on the grid. Now this is made into a coloring problem. 

	% create question enviroment 
	How many \(9 \times 9\) Sudoku puzzles?
	
	Each graph has \(81\) vertices with degree \(20\), so there are \(810\) edges.   
\end{example} 


