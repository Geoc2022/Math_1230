\lecture{9}{2023-03-02}{}

% How proofs go in this class:
% proof is mostly riguous (maybe a confusing element)
% student finds minor problem in the proof maybe on the rigour or the confusing element
% half the audiance then becomes confused
% prof beyna explains that his proof was corret or mostly corect (satifies everyone but the pedantics)
% somone asks somthing about the proof stucture that is relitivly straightforward
% Confusing element gets adressed if not already
% proof is still not tottally rigorus and some people may complain - proof by intimidation may be used to fix the rigour

\begin{note}
	Capacity of any cut gives upper bound on value of any flow. From this we can see the reasoning of why Ford-Fulkerson finds this as a duel problem 
\end{note}

