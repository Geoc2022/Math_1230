\lecture{6}{2023-02-16}{}

\begin{lemma}
	Let \(Q\) be a vertex cover, \(M\) a matching
	Then \(|M| \leq |Q|\) 
\end{lemma}
\begin{proof}
	For each \(e \in M\) there is at least one vertex of \(Q\) incident to \(e\). No two \(e, e^\prime \in M\) share a vertex of \(Q\)
\end{proof}

We can use this to show that if we have a vertex cover that covers all vertices with the same number of a matching we have a maximum matching.

\begin{theorem}[König]
	\label{thm:könig}
	Let \(G\) bipartite, then the max size of the matching is equal to the min size of the vertex cover
\end{theorem} 

Therefore, matching and vertex covers are duel problems

\begin{definition}[M-augmenting-path]
	\label{def:m-augmenting-path}
	A path whose edges alternate between \(M\) and \(E\setminus M\) whose end points are unsaturated.

	% picture spquare and hexagon
\end{definition}  

Local property for max matchings: If there exists an M-agumenting path then M is not maximum

\begin{theorem}[maximum matching]
	\label{thm:maximum matching}
	If there's no M-augmenting path then \(M\)  is maximium
\end{theorem}
\begin{proof}
	\begin{note}
		Suppose \(M\) is not a maximium there exists a \(M^\prime\) s.t. \(|m| < | M^\prime|\) consider the symmetric diffrence \(M \triangle M^\prime\). Here vertices have either degree \(1\) or \(2\) and the graph is aunion of even cycles and paths. 
	\end{note}
	Let \(M\) be a non maximal matching, \(M^\prime\) maximum matching. Consider \(M \triangle M^\prime\) By above this is a union of paths and even cycles. Since \(M\) and \(M^\prime\) share the same size in cycles then some compoent in \(M \triangle M^\prime\) has a path
\end{proof}



