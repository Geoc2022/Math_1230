\lecture{6}{2023-02-16}{}

\begin{lemma}
	Let \(Q\) be a vertex cover, \(M\) a matching
	Then \(|M| \leq |Q|\) 
\end{lemma}
\begin{proof}
	For each \(e \in M\) there is at least one vertex of \(Q\) incident to \(e\). No two \(e, e^\prime \in M\) share a vertex of \(Q\)
\end{proof}

\begin{note}
	We can use this to show that if we have a vertex cover that covers all vertices with the same size as a matching we must have a maximum matching/minimum vertex cover.
\end{note} 

\begin{theorem}[König]
	\label{thm:könig}
	Let \(G\) bipartite, then the max size of the matching is equal to the min size of the vertex cover
\end{theorem} 

\begin{remark}
	Therefore, matching and vertex covers are duel problems
\end{remark}

\begin{definition}[m-augmenting-path]
	\label{def:m-augmenting-path}
	A path whose edges alternate between \(M\) and \(E\setminus M\) whose end points are unsaturated.
\end{definition}  

\begin{theorem}[maximum matching]
	\label{thm:maximum matching}
	Local to Global property: If there's no M-augmenting path then \(M\) is maximum
\end{theorem}
\begin{proof}
	Suppose \(M\) is not a maximum there exists an \(M^\prime\) s.t. \(|m| < | M^\prime|\) consider the symmetric difference \(M \triangle M^\prime\). Here vertices have either degree \(1\) or \(2\) and the graph is a union of even cycles and paths. 

	Let \(M\) be a non-maximal matching, \(M^\prime\) maximum matching. Consider \(M \triangle M^\prime\) By above this is a union of paths and even cycles. Since \(M\) and \(M^\prime\) share the same size in cycles then some component in \(M \triangle M^\prime\) has a path
\end{proof}



