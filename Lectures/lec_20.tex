\lecture{20}{2023-04-25}{}

\subsection{Interoperation of the Laplacian}

\begin{definition}[harmonic]
    \label{def:harmonic}
    Given a function \(f: \R^2 \to \R\) we can apply the Laplacian operator where
    \[
        Lf - f_{x x} + f_{y y} 
    \]
    
    A function is harmonic if \(Lf = 0\) 
\end{definition}  

\begin{proposition}
    If a function is harmonic a point's value is equal to the avg value of the ball around it 
\end{proposition} 

If \(Lf = 0\) then for each \(i\), \(f(v_i) = \frac{1}{\deg (v_i)} \sum_{v_i, v_j \in E} f(v_j)\), which is the avg of \(f\) given neighbors of \(v_i\)

The multiplicity of \(0\) as an eigenvalue of \(L\) is the number of components of \(G\)

