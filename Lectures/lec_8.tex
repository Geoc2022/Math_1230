\lecture{8}{2023-02-28}{}

\section{Connectivity}

\begin{definition}[vertex cut]
	\label{def:vertex cut}
	A vertex cut on a graph \(G\) is a subset \(S \subset V\) s.t. \(G \setminus S\) is disconnected
\end{definition}  

\begin{definition}[vertex connectivity]
	\label{def:vertex connectivity}
	The vertex connectivity, \(\kappa (G)\) , is the min of the vertex cut
	\(\kappa (K_n)\) is not defined 
\end{definition}

\begin{lemma}
	If \(G\) is not complete then \(G\)  has a vertex cut
\end{lemma} 
\begin{proof}
	Let \(G\) not be complete, then there exist \(u, v \in V\)  s.t. \(\{u,v\} \notin E\). Then \(S = V\setminus{u, v}\) is a vertex cut
\end{proof}

\subsection{Connectivity Duel Problem}
Let \(G = (V,E)\). for \(x, y \in V\) define \(\kappa (x,y)\) min size of vertex cut that disconnects \(x\) and \(y\)

\begin{note}
	We see that the points we have to remove are related to paths from \(x\) to \(y\) which have no intersection.
\end{note} 

\begin{definition}[disjoint path]
	\label{def:disjoint path}
	A path is disjoint with another path if there are no interior vertices they share
\end{definition}

\begin{definition}[max pairwise dijoint]
	\label{def:max pairwise dijoint}
	Let \(\lambda (x, y)\) be the max # of pairwise disjoint \(x,y\)-paths  
\end{definition}  
\begin{theorem}[Menger]
	\label{thm:menger}
	Let \(G = (V,E)\), \(x,y \in V\), \(\{x,y\} \notin E\) then 
	\(\kappa (x,y) = \lambda (x,y)\) 
\end{theorem}

\subsection{Max flow, min cut}
\begin{definition}[network]
	\label{def:network}
	Is a tuple \(G,s,t,c\) where \(G = (V, E)\) is a directed graph, \(s,t \in V\) "source" and "terminus", \(c: E \to N\) capacity. 
\end{definition}  

\begin{definition}[flow]
	\label{def:flow}
	A flow on a network is \(f: E \to N\) s.t. \(f(e) \leq c(e)\) and conservation law \(f^{+}(v) = \sum_{e \to v} f(e) = \sum_{v \to e} f(e) = f^{-}(v)\) 
\end{definition}

\begin{definition}[value]
	\label{def:value}
	Value of \(f\)  is \(f^{-}(s)\)
\end{definition}  

The general problem is that given a network what is the max value of a flow

\begin{definition}[cut]
	\label{def:cut}
	A cut is a partition \(V = S \sqcap T\) with \(s \in S\), \(t \in T\)
\end{definition}
\begin{definition}[capacity]
	\label{def:capacity}
	The capacity of a cut \(S, T\) is \(\sum_{e} c(e)\) where \(e\) is an edge from \(S\) to \(T\) 
\end{definition}  

\begin{theorem}[Ford-Fulkerson]
	\label{thm:ford-fulkerson}
	Max value of the flow is equal to the min capacity of a cut
\end{theorem} 


