\lecture{8}{2023-02-28}{}

\section{Connectivity}

\begin{definition}[vertex cut]
	\label{def:vertex cut}
	A vertex cut on a graph \(G\) is a subset \(S \subset V\) s.t. \(G \setminus S\) is disconnected
\end{definition}  

\begin{definition}[vertex connectivity]
	\label{def:vertex connectivity}
	Represented \(\kappa (G)\), the vertex connectivity is the minimum size of a vertex cut of \(G\). For \(K_n\), \(\kappa\) is not defined.
\end{definition}

\begin{lemma}
	If \(G\) is not complete then \(G\) has a vertex cut
\end{lemma} 
\begin{proof}
	Let \(G\) not be complete, then there exist \(u, v \in V\)  s.t. \(\{u,v\} \notin E\). Then \(S = V\setminus{u, v}\) is a vertex cut
\end{proof}

\subsection{Connectivity Duel Problem}
\begin{definition}[vertex connectivity between two vertices]
	\label{def:vertex connectivity between two vertices}
	Let \(G = (V,E)\), for \(x, y \in V\) define \(\kappa (x,y)\) min size of vertex cut that disconnects \(x\) and \(y\)
\end{definition}  

\begin{note}
	We see that the points we have to remove are related to paths from \(x\) to \(y\) which are disjoint.
\end{note} 

\begin{definition}[disjoint path]
	\label{def:disjoint path}
	A path is disjoint with another path if there are no interior vertices they share
\end{definition}

\begin{definition}[max pairwise dijoint]
	\label{def:max pairwise dijoint}
	Let \(\lambda (x, y)\) be the max number of pairwise disjoint \(x,y\)-paths  
\end{definition}  
\begin{theorem}[Menger]
	\label{thm:menger}
	Let \(G = (V,E)\), \(x,y \in V\), \(\{x,y\} \notin E\) then 
	\(\kappa (x,y) = \lambda (x,y)\) 
\end{theorem}
\begin{proof}
	\(\kappa (x,y) \geq \lambda  (x,y)\) is easy

	\(\kappa (x,y) \leq \lambda  (x,y)\):
	
	Let \(S\) be a minimum vertex cut. We want to show that there are \(|S|\) disjoint \((x,y)\) paths  Consider two subgroups \(G_x\) and \(G_y\) where \(G_x\) is the union of all paths from \(x\) to \(S\) (doesn't contain vertices form S) and \(G_y\) is the union of all paths from \(y\) to \(S\) (doesn't contain vertices form S). We have broken up the problem into just working with finding paths to \(S\) from \(x\) and \(y\)

	To form a picture like:
	% https://q.uiver.app/?q=WzAsNyxbNCwxLCJ5Il0sWzIsMCwic18xIl0sWzIsMSwic18yIl0sWzIsMiwic18zIl0sWzAsMSwieCJdLFsxLDAsIlxcYnVsbGV0Il0sWzMsMiwiXFxidWxsZXQiXSxbMCwxLCIiLDAseyJzdHlsZSI6eyJib2R5Ijp7Im5hbWUiOiJzcXVpZ2dseSJ9LCJoZWFkIjp7Im5hbWUiOiJub25lIn19fV0sWzAsMiwiIiwyLHsic3R5bGUiOnsiYm9keSI6eyJuYW1lIjoic3F1aWdnbHkifSwiaGVhZCI6eyJuYW1lIjoibm9uZSJ9fX1dLFswLDMsIiIsMix7InN0eWxlIjp7ImJvZHkiOnsibmFtZSI6InNxdWlnZ2x5In0sImhlYWQiOnsibmFtZSI6Im5vbmUifX19XSxbNCwyLCIiLDAseyJzdHlsZSI6eyJib2R5Ijp7Im5hbWUiOiJzcXVpZ2dseSJ9LCJoZWFkIjp7Im5hbWUiOiJub25lIn19fV0sWzQsMywiIiwwLHsic3R5bGUiOnsiYm9keSI6eyJuYW1lIjoic3F1aWdnbHkifSwiaGVhZCI6eyJuYW1lIjoibm9uZSJ9fX1dLFs0LDUsIiIsMCx7InN0eWxlIjp7ImJvZHkiOnsibmFtZSI6InNxdWlnZ2x5In0sImhlYWQiOnsibmFtZSI6Im5vbmUifX19XSxbNSwyLCIiLDAseyJzdHlsZSI6eyJoZWFkIjp7Im5hbWUiOiJub25lIn19fV0sWzUsMSwiIiwxLHsic3R5bGUiOnsiYm9keSI6eyJuYW1lIjoic3F1aWdnbHkifSwiaGVhZCI6eyJuYW1lIjoibm9uZSJ9fX1dLFswLDYsIiIsMSx7InN0eWxlIjp7ImhlYWQiOnsibmFtZSI6Im5vbmUifX19XSxbNiwzLCIiLDEseyJzdHlsZSI6eyJoZWFkIjp7Im5hbWUiOiJub25lIn19fV0sWzYsMiwiIiwxLHsic3R5bGUiOnsiaGVhZCI6eyJuYW1lIjoibm9uZSJ9fX1dXQ==
\[\begin{tikzcd}
	& \bullet & {s_1} \\
	x && {s_2} && y \\
	&& {s_3} & \bullet
	\arrow[squiggly, no head, from=2-5, to=1-3]
	\arrow[squiggly, no head, from=2-5, to=2-3]
	\arrow[squiggly, no head, from=2-5, to=3-3]
	\arrow[squiggly, no head, from=2-1, to=2-3]
	\arrow[squiggly, no head, from=2-1, to=3-3]
	\arrow[squiggly, no head, from=2-1, to=1-2]
	\arrow[no head, from=1-2, to=2-3]
	\arrow[squiggly, no head, from=1-2, to=1-3]
	\arrow[no head, from=2-5, to=3-4]
	\arrow[no head, from=3-4, to=3-3]
	\arrow[no head, from=3-4, to=2-3]
\end{tikzcd}\]
\end{proof}

\subsection{Max Flow, Min Cut}
\begin{definition}[network]
	\label{def:network}
	Is a tuple \(G,s,t,c\) where \(G = (V, E)\) is a directed graph, \(s,t \in V\) "source" and "terminus", \(c: E \to N\) capacity. 
\end{definition}  

\begin{definition}[flow]
	\label{def:flow}
	A flow on a network is \(f: E \to N\) s.t. \(f(e) \leq c(e)\) and conservation law \(f^{+}(v) = \sum_{e \to v} f(e) = \sum_{v \to e} f(e) = f^{-}(v)\) 
\end{definition}

\begin{definition}[value]
	\label{def:value}
	Value of \(f\) is \(f^{-}(s)\)
\end{definition}  

The general problem is that given a network what is the max value of a flow

\begin{definition}[cut]
	\label{def:cut}
	A cut is a partition \(V = S \sqcap T\) with \(s \in S\), \(t \in T\)
\end{definition}
\begin{definition}[capacity]
	\label{def:capacity}
	The capacity of a cut \(S, T\) is \(\sum_{e} c(e)\) where \(e\) is an edge from \(S\) to \(T\) 
\end{definition}  

\subsection{Ford-Fulkerson}
\begin{theorem}[Ford-Fulkerson]
	\label{thm:ford-fulkerson}
	Max value of the flow is equal to the min capacity of a cut
\end{theorem} 
