\lecture{1}{1/26/2023}{}

\section{Intro}
Graph Theory is important

\section{Fundamental Concepts}
\begin{definition}[graph]
	\label{def:graph}
	A graph is a pair \((V,E)\) where \(V\) is the vertex space and \(E\) is the edge space.
\end{definition}

\begin{example}[petersen graph]
	\label{ex:petersen graph}
	A element subset of \(\{1,2,3,4,5\}\) connected by disjointedness
	\newcommand{\petersengraph}{
	\begin{tikzpicture}[every node/.style={draw,circle,thick}]
		\graph[clockwise, radius=2cm] {subgraph C_n [n=5,name=A] };
		\graph[clockwise, radius=1cm] {subgraph I_n [n=5,name=B] };
	  
		\foreach \i [evaluate={\j=int(mod(\i+2+4,5)+1)}]% using Paul Gaborit's optimisation
		   in {1,2,3,4,5}{
		  \draw (A \i) -- (B \i);
		  \draw (B \j) -- (B \i);
		}
	\end{tikzpicture}
	}
	\[
		\petersengraph
	\]
\end{example}

\begin{note}
	In this course we are excluding multiple edges and self loops
\end{note}

\begin{definition}[vertex degrees]
	\label{def:vertex degrees}
	Let, \(G = (V, E)\), \(v \in V\), \(e \in E\)
	\(e\) are incident if \(v \in e\) i.e. \(v\) is an endpoint of \(e\)  
\end{definition}

\begin{lemma}
	\[
		\sum_{v \in V} deg(v) = \sum_{v \in V}\sum_{v \in e} 1 = \sum_{v \in e} \sum_{v \in V} 1 = 2 |E|  
	\]
\end{lemma}
\begin{proof}
	Every edge has two vertices
\end{proof}

\begin{definition}[complete graph]
	\label{def:complete graph}
	Represented \(K_n\), the graph has \(V = {1 \ldots n}\) and all possible edges
	\[
		|E| = \frac{n(n-1)}{2} = nCr(n, 2)
	\]
\end{definition}

\begin{definition}[isomorphic]
	\label{def:isomorphic}
	Graphs \(G_1\) and \(G_2\) are isomorphic if there exists a bijection, \(f: v_1 \to v_2\), s.t. \(\{u, v\} \in E_1 \iff \{f(u), f(v)\}\) 
\end{definition}

\subsection{Connectivity}
\begin{definition}[connected]
	\label{def:connected}
	\(u\) and \(v\) are connected if there exists a path from \(u\) to \(v\)
	A graph is connected when all vertices are connected   
\end{definition}

