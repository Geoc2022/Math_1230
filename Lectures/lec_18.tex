\lecture{18}{2023-04-18}{}

\subsection{Erdos-Renyi Random Graphs and Phase Transitions}

\begin{definition}[labled n vertex graphs]
    \label{def:labled n vertex graphs}
    Let \(\Omega_{n,p} = \{\text{labeled \(n\) vertex graphs}\}\)  with probability function given by flipping biased coin for each edge where an edge occurs with \(p\) probability.
\end{definition}

\begin{definition}[phase transition probability function]
    \label{def:phase transition probability function}
    We can write \(P(G) = p^{|E|} (1- p)^{{n \choose 2} - |E|}\) 
\end{definition}  

\begin{example}[threshold function]
    \label{ex:threshold function}
    \begin{enumerate}
			\item[]
        \item \(t(n) = \ln (n) / n\) is the threshold for connectedness
        \item \(t(n) = \sqrt{2 \ln(n) / n}\) is the threshold for a diameter of \(2\)
        \item \(p(n) = o(1 / n)\) (\(p < c / n\) for \(n\) sufficiently large) then \(G\) is a forest  
        \item \(p(n) ~ c / n\) where \(0 < c < 1\) have cycle of any size
        \item \(p(n) ~ c / n\) where \(1 < c\) has large component
    \end{enumerate}
\end{example} 

\begin{remark}
    Properties that are monotone, i.e. adding edges to the graph doesn't change the property (ex. connectedness)
\end{remark}

\subsection{Rado's Graph}

\begin{definition}[rado's graph]
    \label{def:rado's graph}
    Let \(V = \N\) and \(\exists {i, j} \in E \iff \text{the \(i\)th digit of \(j\) is odd}\)

    \emph{Properties}
    \begin{enumerate}
        \item Robustness - any finite edge/vertex cut does not destroy \(R\)
        \item \(R \cong \bar{R}\)
        \item Universal = any finite/countable \(G\) is induced subgraph of R
        \item Homogeneity - Every isomerism between subgraphs  if \(R\) extension of \(R\) 
    \end{enumerate}
\end{definition}  

\begin{definition}[extension property]
    \label{def:extension property}
    Given \(u_1 \cdots u_p\) and \(v_1 \cdots v_q\) distinct  \(\exists z\) adjacent to all \(u\)s and none of the \(v\)s 
\end{definition}  

\begin{theorem}
    A = \(\{G \in \Omega_{\infty} \text{isomophic to \(R\)}\} \in \Omega_{\infty}\) 
\end{theorem} 
\begin{proof}
    \emph{\(B = \{G \text{ has the extension property }\} \implies P(B) = 1\)}
        Let us compute the probability that the extension property fails
        
        Given \(u_1 \cdots u_p\), \(v_1 \cdots v_q\), \(z_1 \cdots z_r\), then the probability is \({p + q + r \choose p + q} (1 - \frac{1}{2^{p + q}})^{r}\) 
\end{proof}
