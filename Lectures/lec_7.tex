\lecture{7}{2023-02-23}{}

\begin{definition}[stable matching]
    \label{def:stable matching}
    A matching is stable if 
    % https://q.uiver.app/?q=WzAsNCxbMCwwLCJ4Il0sWzEsMSwieVxccHJpbWUiXSxbMSwwLCJ4XFxwcmltZSJdLFswLDEsInkiXSxbMCwzLCIiLDAseyJzdHlsZSI6eyJoZWFkIjp7Im5hbWUiOiJub25lIn19fV0sWzIsMSwiIiwyLHsic3R5bGUiOnsiaGVhZCI6eyJuYW1lIjoibm9uZSJ9fX1dLFswLDEsIiIsMCx7InN0eWxlIjp7ImJvZHkiOnsibmFtZSI6ImRvdHRlZCJ9LCJoZWFkIjp7Im5hbWUiOiJub25lIn19fV1d
\[\begin{tikzcd}
	x & x\prime \\
	y & y\prime
	\arrow[no head, from=1-1, to=2-1]
	\arrow[no head, from=1-2, to=2-2]
	\arrow[dotted, no head, from=1-1, to=2-2]
\end{tikzcd}\]
    then either \(x\) prefers \(y\) to \(y^\prime\) or \(y^\prime\) prefers \(x^\prime\) to x
\end{definition}  

\begin{theorem}[Gale-Shapely]
    \label{thm:gale-shapely}
    Stable matchings always exist in bipartite graphs
\end{theorem} 
\begin{proof}
    We preform an algorithm to construct a stable matching:
	\begin{enumerate}[align=left]
		\item[\textbf{Round 1:}] 
		\begin{enumerate}[align=left]
			\item[] 
			\item[] Each \(x \in X\) proposes to top choice
			\item[] Each \(x \in X\) proposes to top choice
		\end{enumerate}
		\item[\textbf{Subsequent Rounds:}] Each unmatched \(x\) proposes to top choice not yet proposed to. Each \(y\) accepts the best proposal possible breaking an engagement
		\item[\textbf{Repeat}]
	\end{enumerate}
		
    \emph{Features}
    \begin{enumerate}[align=left]
        \item Algorithm stops after \(<|E|\) rounds
        \item End matching is stable
        \item Proposers get the best match among all stable matchings. Proposed get worse.
    \end{enumerate}
\end{proof}
