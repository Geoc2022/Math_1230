\lecture{3}{2023-02-02}{}

\begin{remark}
	Algorithmic Bipartite Testing
	\begin{itemize}
		\item Brute force: \(2^{| V |} \cdot | E |\) 
		\item Proof Algorithm: where we color the vertices and check the edges \(| V | + | E | \) 
	\end{itemize}
\end{remark}

\begin{remark}[local to global results]
	\label{remark:local to global results}
	Global properties always lead to local results (ex. bipartite implies no odd cycles). In West it is called "TONCAS"
\end{remark}

\subsection{Eulerian Graphs}

\begin{definition}[euler tour]
	\label{def:euler tour}
	A closed walk that visits each edge of a graph exactly once
\end{definition}

\begin{definition}[eulerian]
	\label{def:eulerian}
	A graph with an euler tour is Eulerian.
\end{definition}

\begin{proposition}[eulerian local property]
	\label{prop:eulerian local property}
	A graph is Eulerian iff every vertex has even degree.
\end{proposition}

\begin{example}[seven bridges of königsberg]
	\label{ex:seven bridges of königsberg}
	% https://q.uiver.app/?q=WzAsNCxbMCwxLCJcXGJ1bGxldCJdLFsyLDEsIlxcYnVsbGV0Il0sWzEsMCwiXFxidWxsZXQiXSxbMSwyLCJcXGJ1bGxldCJdLFswLDFdLFswLDIsIiIsMix7ImN1cnZlIjoxfV0sWzIsMV0sWzEsM10sWzMsMCwiIiwyLHsiY3VydmUiOi0xfV0sWzIsMCwiIiwyLHsiY3VydmUiOjF9XSxbMCwzLCIiLDIseyJjdXJ2ZSI6LTF9XV0=
\[\begin{tikzcd}
	& \bullet \\
	\bullet && \bullet \\
	& \bullet
	\arrow[from=2-1, to=2-3]
	\arrow[curve={height=6pt}, from=2-1, to=1-2]
	\arrow[from=1-2, to=2-3]
	\arrow[from=2-3, to=3-2]
	\arrow[curve={height=-6pt}, from=3-2, to=2-1]
	\arrow[curve={height=6pt}, from=1-2, to=2-1]
	\arrow[curve={height=-6pt}, from=2-1, to=3-2]
\end{tikzcd}\]
	Using a graph to model the seven bridges of Königsberg shows that an euler tour is not possible.
\end{example} 

\begin{note}
	This is another example of local and global properties where the parity of the degree is the local property and Eulerian is the global property
\end{note} 

\section{Trees}
\begin{definition}[tree]
	\label{def:tree}
	A tree is a \nameref{def:connected} and \hyperref[def:cycle]{acyclic} graph
\end{definition}

\begin{example}
	% https://q.uiver.app/?q=WzAsOSxbMCwxLCJcXGJ1bGxldCJdLFsxLDEsIlxcYnVsbGV0Il0sWzIsMCwiXFxidWxsZXQiXSxbMiwyLCJcXGJ1bGxldCJdLFszLDEsIlxcYnVsbGV0Il0sWzMsMiwiXFxidWxsZXQiXSxbNCwwLCJcXGJ1bGxldCJdLFs0LDEsIlxcYnVsbGV0Il0sWzUsMSwiXFxidWxsZXQiXSxbMCwxLCIiLDAseyJzdHlsZSI6eyJoZWFkIjp7Im5hbWUiOiJub25lIn19fV0sWzEsMiwiIiwwLHsic3R5bGUiOnsiaGVhZCI6eyJuYW1lIjoibm9uZSJ9fX1dLFsxLDMsIiIsMCx7InN0eWxlIjp7ImhlYWQiOnsibmFtZSI6Im5vbmUifX19XSxbMyw0LCIiLDAseyJzdHlsZSI6eyJoZWFkIjp7Im5hbWUiOiJub25lIn19fV0sWzMsNSwiIiwwLHsic3R5bGUiOnsiaGVhZCI6eyJuYW1lIjoibm9uZSJ9fX1dLFs0LDYsIiIsMCx7InN0eWxlIjp7ImhlYWQiOnsibmFtZSI6Im5vbmUifX19XSxbNCw3LCIiLDAseyJzdHlsZSI6eyJoZWFkIjp7Im5hbWUiOiJub25lIn19fV0sWzcsOCwiIiwwLHsic3R5bGUiOnsiaGVhZCI6eyJuYW1lIjoibm9uZSJ9fX1dXQ==
\[\begin{tikzcd}
	&& \bullet && \bullet \\
	\bullet & \bullet && \bullet & \bullet & \bullet \\
	&& \bullet & \bullet
	\arrow[no head, from=2-1, to=2-2]
	\arrow[no head, from=2-2, to=1-3]
	\arrow[no head, from=2-2, to=3-3]
	\arrow[no head, from=3-3, to=2-4]
	\arrow[no head, from=3-3, to=3-4]
	\arrow[no head, from=2-4, to=1-5]
	\arrow[no head, from=2-4, to=2-5]
	\arrow[no head, from=2-5, to=2-6]
\end{tikzcd}\]
\end{example}

\begin{definition}[leaf]
	\label{def:leaf}
	A node of degree \(1\) 
\end{definition}

\begin{lemma}
	Any connected subgraph of a tree is a tree 
\end{lemma}
\begin{proof}
	By contradiction:

	Assume that the connected subgraph is not a tree, then the subgraph has a cycle, therefore the graph has a cycle, but trees are acyclic. Therefore, contradiction.
\end{proof}

\begin{lemma}
	A tree with \(n\) vertices has \(n-1\) edges
\end{lemma}
\begin{proof}
	By induction:
	
	\emph{Base Case:} There are \(0\) edges in a \(1\) vertex tree

	\emph{Inductive Step:} Suppose this holds for \(n\).
	Let \(T\) be a tree with \(n+1\) vertices, then by removing one vertex, which must be a leaf — removing a non-leaf would make the graph unconnected, you remove one edge and the graph becomes the \(n\) case.
	Therefore, the \(n+1\) tree has \(1\) more edge than an \(n\) tree.
\end{proof}

\subsection{Spanning Trees}
\begin{definition}[spanning tree]
	\label{def:spanning tree}
	A spanning tree (ST) is a subgraph of a \nameref{def:connected} that is a \nameref{def:tree} that contains all vertices of the original graph
\end{definition}

\begin{lemma}
	There is a spanning tree in every connected graph
\end{lemma}
\begin{proof}
	By contradiction: \\
	Assume \(G\) is a connected graph with no ST.
	Let \(T\) be a connected subgraph of \(G\)  that has the same vertices as \(G\) with the smallest number of edges. Since \(T\) is not a tree, it does not have a cycle. However, \(T\) containing a cycle would imply that \(T\) is not a subgraph that has the smallest number of edges.
	Therefore, contradiction test 
\end{proof}

\begin{proposition}
    With a graph \(G\) where \(T, T^\prime \subset G\) s.t. \(e \in T, e \notin T^\prime\). Then there is edge \(e^\prime \in T^\prime \) s.t. \(T - e + e^\prime\)  is a spanning tree.
\end{proposition}
\begin{proof}
    \(T\) is a tree \(\implies\) \(T - e\) is disconnected. \(T, T^\prime\) have the same vertex set \(\implies \exists e \in T^\prime\) that connects two components of \(T - e\) Lets claim that \(T - e + e^\prime\) is a spanning tree. It has the same number of edges and is connected, so it's a tree, and is spanning since it connects all edges. 
\end{proof}
