\lecture{14}{2023-03-21}{}

\subsection{Art Gallery Problem}
\begin{question}
    Given a room, how many light sources can fill the entire space?
    % https://q.uiver.app/?q=WzAsMTAsWzAsMiwiXFxidWxsZXQiXSxbMCw1LCJcXGJ1bGxldCJdLFsyLDUsIlxcYnVsbGV0Il0sWzMsMywiXFxidWxsZXQiXSxbNSwzLCJcXGJ1bGxldCJdLFs1LDEsIlxcYnVsbGV0Il0sWzQsMCwiXFxidWxsZXQiXSxbNCwxLCJcXGJ1bGxldCJdLFsyLDEsIlxcYnVsbGV0Il0sWzIsMiwiXFxidWxsZXQiXSxbMCwxLCIiLDAseyJzdHlsZSI6eyJoZWFkIjp7Im5hbWUiOiJub25lIn19fV0sWzEsMiwiIiwwLHsic3R5bGUiOnsiaGVhZCI6eyJuYW1lIjoibm9uZSJ9fX1dLFsyLDMsIiIsMCx7InN0eWxlIjp7ImhlYWQiOnsibmFtZSI6Im5vbmUifX19XSxbMyw0LCIiLDAseyJzdHlsZSI6eyJoZWFkIjp7Im5hbWUiOiJub25lIn19fV0sWzQsNSwiIiwwLHsic3R5bGUiOnsiaGVhZCI6eyJuYW1lIjoibm9uZSJ9fX1dLFs1LDYsIiIsMCx7InN0eWxlIjp7ImhlYWQiOnsibmFtZSI6Im5vbmUifX19XSxbNiw3LCIiLDAseyJzdHlsZSI6eyJoZWFkIjp7Im5hbWUiOiJub25lIn19fV0sWzcsOCwiIiwwLHsic3R5bGUiOnsiaGVhZCI6eyJuYW1lIjoibm9uZSJ9fX1dLFs4LDksIiIsMCx7InN0eWxlIjp7ImhlYWQiOnsibmFtZSI6Im5vbmUifX19XSxbOSwwLCIiLDAseyJzdHlsZSI6eyJoZWFkIjp7Im5hbWUiOiJub25lIn19fV1d
\[\begin{tikzcd}
	&&&& \bullet \\
	&& \bullet && \bullet & \bullet \\
	\bullet && \bullet \\
	&&& \bullet && \bullet \\
	\\
	\bullet && \bullet
	\arrow[no head, from=3-1, to=6-1]
	\arrow[no head, from=6-1, to=6-3]
	\arrow[no head, from=6-3, to=4-4]
	\arrow[no head, from=4-4, to=4-6]
	\arrow[no head, from=4-6, to=2-6]
	\arrow[no head, from=2-6, to=1-5]
	\arrow[no head, from=1-5, to=2-5]
	\arrow[no head, from=2-5, to=2-3]
	\arrow[no head, from=2-3, to=3-3]
	\arrow[no head, from=3-3, to=3-1]
\end{tikzcd}\]
\end{question}
\begin{theorem}[Art Gallery Thm]
    \label{thm:art gallery thm}
    You only need \(\big\lfloor \frac{n}{3} \big\rfloor\) for a polygon with \(n\) sides
\end{theorem}
\begin{proof}[5 sided case]
    Given a \(5\) sided polygon, consider the convex hull of the vertices. 
    
    The cases for the convex hull can be either \(5,4,3\). For these cases, if you split the polygon into traingles there exists a vertex where it is incident to every triangle. And the solution is that you put the source near this vertex.

% https://q.uiver.app/?q=WzAsMTUsWzYsMCwiXFxidWxsZXQiXSxbNiw0LCJcXGJ1bGxldCJdLFs4LDIsIioiXSxbMTAsNCwiXFxidWxsZXQiXSxbMTAsMCwiXFxidWxsZXQiXSxbMiwwLCIqIl0sWzAsMiwiXFxidWxsZXQiXSxbMCw0LCJcXGJ1bGxldCJdLFs0LDQsIlxcYnVsbGV0Il0sWzQsMiwiXFxidWxsZXQiXSxbMTQsMCwiKiJdLFsxMiw0LCJcXGJ1bGxldCJdLFsxMywzLCJcXGJ1bGxldCJdLFsxNSwzLCJcXGJ1bGxldCJdLFsxNiw0LCJcXGJ1bGxldCJdLFswLDEsIiIsMCx7InN0eWxlIjp7ImhlYWQiOnsibmFtZSI6Im5vbmUifX19XSxbMSwyLCIiLDAseyJzdHlsZSI6eyJoZWFkIjp7Im5hbWUiOiJub25lIn19fV0sWzIsMywiIiwwLHsic3R5bGUiOnsiaGVhZCI6eyJuYW1lIjoibm9uZSJ9fX1dLFszLDQsIiIsMCx7InN0eWxlIjp7ImhlYWQiOnsibmFtZSI6Im5vbmUifX19XSxbNCwwLCIiLDAseyJzdHlsZSI6eyJoZWFkIjp7Im5hbWUiOiJub25lIn19fV0sWzUsNiwiIiwwLHsic3R5bGUiOnsiaGVhZCI6eyJuYW1lIjoibm9uZSJ9fX1dLFs2LDcsIiIsMCx7InN0eWxlIjp7ImhlYWQiOnsibmFtZSI6Im5vbmUifX19XSxbNyw4LCIiLDAseyJzdHlsZSI6eyJoZWFkIjp7Im5hbWUiOiJub25lIn19fV0sWzgsOSwiIiwwLHsic3R5bGUiOnsiaGVhZCI6eyJuYW1lIjoibm9uZSJ9fX1dLFs5LDUsIiIsMCx7InN0eWxlIjp7ImhlYWQiOnsibmFtZSI6Im5vbmUifX19XSxbMTAsMTEsIiIsMCx7InN0eWxlIjp7ImhlYWQiOnsibmFtZSI6Im5vbmUifX19XSxbMTEsMTIsIiIsMCx7InN0eWxlIjp7ImhlYWQiOnsibmFtZSI6Im5vbmUifX19XSxbMTIsMTMsIiIsMCx7InN0eWxlIjp7ImhlYWQiOnsibmFtZSI6Im5vbmUifX19XSxbMTMsMTQsIiIsMCx7InN0eWxlIjp7ImhlYWQiOnsibmFtZSI6Im5vbmUifX19XSxbMTQsMTAsIiIsMCx7InN0eWxlIjp7ImhlYWQiOnsibmFtZSI6Im5vbmUifX19XSxbMSwzLCIiLDEseyJzdHlsZSI6eyJib2R5Ijp7Im5hbWUiOiJkYXNoZWQifSwiaGVhZCI6eyJuYW1lIjoibm9uZSJ9fX1dLFsxMSwxNCwiIiwxLHsic3R5bGUiOnsiYm9keSI6eyJuYW1lIjoiZGFzaGVkIn0sImhlYWQiOnsibmFtZSI6Im5vbmUifX19XSxbNSw3LCIiLDEseyJzdHlsZSI6eyJib2R5Ijp7Im5hbWUiOiJkb3R0ZWQifSwiaGVhZCI6eyJuYW1lIjoibm9uZSJ9fX1dLFs1LDgsIiIsMSx7InN0eWxlIjp7ImJvZHkiOnsibmFtZSI6ImRvdHRlZCJ9LCJoZWFkIjp7Im5hbWUiOiJub25lIn19fV0sWzAsMiwiIiwxLHsic3R5bGUiOnsiYm9keSI6eyJuYW1lIjoiZG90dGVkIn0sImhlYWQiOnsibmFtZSI6Im5vbmUifX19XSxbNCwyLCIiLDEseyJzdHlsZSI6eyJib2R5Ijp7Im5hbWUiOiJkb3R0ZWQifSwiaGVhZCI6eyJuYW1lIjoibm9uZSJ9fX1dLFsxMCwxMiwiIiwxLHsic3R5bGUiOnsiYm9keSI6eyJuYW1lIjoiZG90dGVkIn0sImhlYWQiOnsibmFtZSI6Im5vbmUifX19XSxbMTAsMTMsIiIsMSx7InN0eWxlIjp7ImJvZHkiOnsibmFtZSI6ImRvdHRlZCJ9LCJoZWFkIjp7Im5hbWUiOiJub25lIn19fV1d
\[\adjustbox{scale=.60,center}{
\begin{tikzcd}[scale=0.50]
	&& {*} &&&& \bullet &&&& \bullet &&&& {*} \\
	\\
	\bullet &&&& \bullet &&&& {*} \\
	&&&&&&&&&&&&& \bullet && \bullet \\
	\bullet &&&& \bullet && \bullet &&&& \bullet && \bullet &&&& \bullet
	\arrow[no head, from=1-7, to=5-7]
	\arrow[no head, from=5-7, to=3-9]
	\arrow[no head, from=3-9, to=5-11]
	\arrow[no head, from=5-11, to=1-11]
	\arrow[no head, from=1-11, to=1-7]
	\arrow[no head, from=1-3, to=3-1]
	\arrow[no head, from=3-1, to=5-1]
	\arrow[no head, from=5-1, to=5-5]
	\arrow[no head, from=5-5, to=3-5]
	\arrow[no head, from=3-5, to=1-3]
	\arrow[no head, from=1-15, to=5-13]
	\arrow[no head, from=5-13, to=4-14]
	\arrow[no head, from=4-14, to=4-16]
	\arrow[no head, from=4-16, to=5-17]
	\arrow[no head, from=5-17, to=1-15]
	\arrow[dashed, no head, from=5-7, to=5-11]
	\arrow[dashed, no head, from=5-13, to=5-17]
	\arrow[dotted, no head, from=1-3, to=5-1]
	\arrow[dotted, no head, from=1-3, to=5-5]
	\arrow[dotted, no head, from=1-7, to=3-9]
	\arrow[dotted, no head, from=1-11, to=3-9]
	\arrow[dotted, no head, from=1-15, to=4-14]
	\arrow[dotted, no head, from=1-15, to=4-16]
\end{tikzcd}}\]

Here are each of the convex hull cases (\(5, 4, 3\)) with the convex hull with long dashed lines and the triangle sections separated by the dotted lines.
\end{proof}

\subsection{Map Coloring}
\begin{question}
    Given a map how many colors are needed to color all the regions s.t. adjacent regions have different colors
\end{question}

\begin{observe}
    The dual graph of them map is planar since we can move the vertices to the border, the borders do not intersect, and by Fary's.
\end{observe}

\begin{theorem}[4 Color]
    \label{thm:4 color}
    Every planar graph is \(4\) colorable
\end{theorem} 

\begin{theorem}[6 Color]
    \label{thm:6 Color}
    Every planar graph is \(6\) colorable
\end{theorem}
\begin{proof}
    A planar graph has a vertex of degree \(\leq 5\) by Euler's Formula. By induction, \(6\) color \(G\setminus v\). If we remove \(v\) we get another planer graph that has a vertex of degree \(\leq 5\) By applying this repeatedly we get an ordering \(v_1 \cdots v_n\). Now we can be greedy color by the order \(v_n \cdots v_i\) where \(v_i\) has degree \(\leq 5\) 
\end{proof}

\subsection{Planar Regular Graph}
\begin{definition}[regular planar]
    \label{def:regular planar}
    A connected graph is called planar regular if it has an embedding where the vertices all have the same number of sides, which implies the duel is regular.
\end{definition}  

\begin{question}
    What are all planer regular graphs?
\end{question}

\(G\) is k regular \(2 |E| = k |V|\)
planar regular \(2 |E| = l |F|\) 

and using Euler we get \(\frac{1}{k} + \frac{1}{l} = \frac{1}{|E|} + \frac{1}{2} > \frac{1}{2}\) 