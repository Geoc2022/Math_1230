\documentclass[11pt]{article}

\author{Math 123}
\date{Due April 7, 2023 by midnight} 
\title{Homework 8}

\usepackage{graphicx,xypic}
\usepackage{amsthm}
\usepackage{amsmath,amssymb}
\usepackage{amsfonts}
\usepackage{xcolor}
\usepackage[margin=1in]{geometry}
\usepackage[shortlabels]{enumitem}
\newtheorem{problem}{Problem}
\renewcommand*{\proofname}{{\color{blue}Solution}}


\usepackage{fancyhdr}
\pagestyle{fancy}
\rhead{Math 123, Homework 8}

\setlength{\parindent}{0pt}
\setlength{\parskip}{1.25ex}

\input{../tikz_import.tex}

\begin{document}

\maketitle

% You are required to put your name here:
{\bf\Large Name: George Chemmala} 


\vspace{.3in}
Topics covered: planar graphs, Kuratowski's theorem

Instructions: 
\begin{itemize}
\item This assignment must be submitted on Gradescope by the due date. 
\item If you collaborate with other students (which is encouraged!), please mention this somewhere on the assignment. 
\item If you are stuck, please ask for help (from me, a TA, a classmate). Use Campuswire!  
\item You may freely use any fact proved in class. In general, you should provide proof for facts used that were not proved in class. 
\item Please restrict your solution to each problem to a single page. Usually solutions can be even shorter than that. If your solution is very long, you should think more about how to express it concisely.
\end{itemize}



\pagebreak 


\begin{problem}
Prove that the Petersen graph is not planar using Euler's formula (do not use Kuratowski's theorem). \footnote{Hint: you will need to use a problem from HW1.} 
\end{problem}

\begin{proof}
Let \(P\) represent the Petersen graph

By contradiction:
	
	Assume there exists an embedding \(P\) in \(\mathbb{R}^2\) then by Euler's Formula 
	\begin{eqnarray*}
		|F| = 2 - |V| + |E| \\
		|F| = 2 - 10 + 15 \\
		|F| = 2 - 5 + 10 \\
		|F| = 7 \\
	\end{eqnarray*}

	But each edge has at most \(2\) faces and each face has \(\geq 5\) sides (the minimum cycle size in \(P\) is \(5\)) therefore \(2 |E| \geq 3 |F|\). However, \(2 \cdot 15 = 30 \geq 35 = 5 \cdot 7\) is a contradiction; therefore \(P\) is not planer.
\end{proof}

\pagebreak

\begin{problem}
Let $G$ be a connected graph embedded in the plane. Prove that $G$ is bipartite if and only if every region of $\mathbb R^2\setminus G$ has an even number of sides. \footnote{Note: the boundary of a region does \underline{not} necessarily correspond to a cycle in the graph. Make sure you understand why!} \footnote{Hint: for one direction, use induction on the number of regions of $\mathbb R^2\setminus G$.} 
\end{problem}

\begin{proof}
	A bipartite graph contains no closed cycle of odd length, so it contains no faces with odd sides. 
	
	Reverse direction: Suppose every face has an even number of sides.

	By induction:
	\begin{enumerate}[align=left]
		\item[\emph{Base Case:}] (\(|F| = 1\))
	
		If \(G\) has only one face then it is a tree, and so bipartite
		\item[\emph{Inductive Step:}]  
	
		Fix edge \(e\) on a cycle in \(G\), this edge borders faces \(F_1, F_2\). By removing \(e\), we remove one side from \(F_1, F_2\) and combine them to have a bigger face \(F_3\) that has the sum of sides of \(F_1\) and \(F_2\) minus \(2\), so it has even sides. 
		
		By inductive hypothesis, we know that \(G \setminus e\) is bipartite. Since the vertices \(e\) is connected to have an odd walk between them, adding \(e\) maintains that \(G\) is bipartite. 
	\end{enumerate} 
\end{proof}


\pagebreak

\begin{problem}
Prove that the complement of a planar graph with at least $11$ vertices is nonplanar.\footnote{Hint: your solution should be very short.} Give an example of a planar graph with $8$ vertices whose complement is also planar.\footnote{Hint: in fact you can find one that is self-complementary. This should simplify your search.}
\end{problem}

\begin{proof}
	Let \(T(n)\) represent the \(n\)th triangle number.

	\begin{eqnarray*}
		|F| = 2 - |V| + |E| \\
		|F| = 2 - 11 + |E|
	\end{eqnarray*}

	Since \(2|E| = 3|F|\):
	\begin{eqnarray*}
		2|E| \geq 6 - 3(11) + 3|E| \\
		2|E| \geq 6 - 33 + 3|E| \\
		2|E| \geq -27 + 3|E| \\
		-|E| \geq -27 \\
		|E| \leq 27 \\
	\end{eqnarray*}

	\(G\) being planar implies that \(|E| \leq 27\). Therefore, \(\overline{G}\) has greater than \(T(11 - 1) = 55 - 27 = 28\) edges (\(K_{11}\) has \(T(11 - 1) = 55\) edges), which is more than \(27\), which means it's not planar.

\[\begin{tikzcd}
	% https://q.uiver.app/?q=WzAsOCxbMCwyLCJcXGJ1bGxldCJdLFsxLDIsIlxcYnVsbGV0Il0sWzIsMiwiXFxidWxsZXQiXSxbMywyLCJcXGJ1bGxldCJdLFs0LDAsIlxcYnVsbGV0Il0sWzAsMCwiXFxidWxsZXQiXSxbNCw0LCJcXGJ1bGxldCJdLFswLDQsIlxcYnVsbGV0Il0sWzQsMCwiIiwwLHsic3R5bGUiOnsiaGVhZCI6eyJuYW1lIjoibm9uZSJ9fX1dLFs0LDEsIiIsMix7InN0eWxlIjp7ImhlYWQiOnsibmFtZSI6Im5vbmUifX19XSxbNCwyLCIiLDIseyJzdHlsZSI6eyJoZWFkIjp7Im5hbWUiOiJub25lIn19fV0sWzQsMywiIiwyLHsic3R5bGUiOnsiaGVhZCI6eyJuYW1lIjoibm9uZSJ9fX1dLFs2LDMsIiIsMCx7InN0eWxlIjp7ImhlYWQiOnsibmFtZSI6Im5vbmUifX19XSxbNiwyLCIiLDEseyJzdHlsZSI6eyJoZWFkIjp7Im5hbWUiOiJub25lIn19fV0sWzYsMSwiIiwxLHsic3R5bGUiOnsiaGVhZCI6eyJuYW1lIjoibm9uZSJ9fX1dLFs2LDAsIiIsMSx7InN0eWxlIjp7ImhlYWQiOnsibmFtZSI6Im5vbmUifX19XSxbNSw0LCIiLDEseyJzdHlsZSI6eyJoZWFkIjp7Im5hbWUiOiJub25lIn19fV0sWzQsNiwiIiwxLHsic3R5bGUiOnsiaGVhZCI6eyJuYW1lIjoibm9uZSJ9fX1dLFs2LDcsIiIsMSx7InN0eWxlIjp7ImhlYWQiOnsibmFtZSI6Im5vbmUifX19XSxbMCwxLCIiLDEseyJzdHlsZSI6eyJoZWFkIjp7Im5hbWUiOiJub25lIn19fV0sWzEsMiwiIiwxLHsic3R5bGUiOnsiaGVhZCI6eyJuYW1lIjoibm9uZSJ9fX1dLFsyLDMsIiIsMSx7InN0eWxlIjp7ImhlYWQiOnsibmFtZSI6Im5vbmUifX19XV0=
	\bullet &&&& \bullet \\
	\\
	\bullet & \bullet & \bullet & \bullet \\
	\\
	\bullet &&&& \bullet
	\arrow[no head, from=1-5, to=3-1]
	\arrow[no head, from=1-5, to=3-2]
	\arrow[no head, from=1-5, to=3-3]
	\arrow[no head, from=1-5, to=3-4]
	\arrow[no head, from=5-5, to=3-4]
	\arrow[no head, from=5-5, to=3-3]
	\arrow[no head, from=5-5, to=3-2]
	\arrow[no head, from=5-5, to=3-1]
	\arrow[no head, from=1-1, to=1-5]
	\arrow[no head, from=1-5, to=5-5]
	\arrow[no head, from=5-5, to=5-1]
	\arrow[no head, from=3-1, to=3-2]
	\arrow[no head, from=3-2, to=3-3]
	\arrow[no head, from=3-3, to=3-4]
\end{tikzcd}\]
	This graph was creating using the algorithm from the bonus problem in HW1 
\end{proof}

\pagebreak

\begin{problem}
Let $G_n$ be the graph with vertices $v_1,\ldots,v_n$ and an edge between $v_i$ and $v_j$ whenever $|i-j|\le 3$. Prove that $G_n$ is a maximal planar graph.
\end{problem}

\begin{proof}
	From class we know that \(G\) is maximal planar \(\iff  G\) is planer and \(|E| = 3|V| - 6\)

	We can see that \(G\) is planer by constuction:
	\[\begin{tikzcd}
		% https://q.uiver.app/?q=WzAsMTgsWzEsMCwidl8yIl0sWzAsMiwidl8zIl0sWzIsMiwidl8xIl0sWzUsMiwidl8xIl0sWzQsMCwidl8yIl0sWzMsMiwidl8zIl0sWzQsMSwidl80Il0sWzMsNiwidl8xIl0sWzEsMywidl8yIl0sWzAsNiwidl8zIl0sWzIsNSwidl80Il0sWzEsNCwidl81Il0sWzcsNiwidl8xIl0sWzUsMywidl8yIl0sWzQsNiwidl8zIl0sWzYsNSwidl80Il0sWzUsNCwidl81Il0sWzUsNSwidl82Il0sWzAsMSwiIiwwLHsic3R5bGUiOnsiaGVhZCI6eyJuYW1lIjoibm9uZSJ9fX1dLFsxLDIsIiIsMCx7InN0eWxlIjp7ImhlYWQiOnsibmFtZSI6Im5vbmUifX19XSxbMiwwLCIiLDAseyJzdHlsZSI6eyJoZWFkIjp7Im5hbWUiOiJub25lIn19fV0sWzMsNCwiIiwwLHsic3R5bGUiOnsiaGVhZCI6eyJuYW1lIjoibm9uZSJ9fX1dLFs0LDUsIiIsMCx7InN0eWxlIjp7ImhlYWQiOnsibmFtZSI6Im5vbmUifX19XSxbNSwzLCIiLDAseyJzdHlsZSI6eyJoZWFkIjp7Im5hbWUiOiJub25lIn19fV0sWzMsNiwiIiwwLHsic3R5bGUiOnsiaGVhZCI6eyJuYW1lIjoibm9uZSJ9fX1dLFs0LDYsIiIsMCx7InN0eWxlIjp7ImhlYWQiOnsibmFtZSI6Im5vbmUifX19XSxbNiw1LCIiLDAseyJzdHlsZSI6eyJoZWFkIjp7Im5hbWUiOiJub25lIn19fV0sWzcsOCwiIiwwLHsic3R5bGUiOnsiaGVhZCI6eyJuYW1lIjoibm9uZSJ9fX1dLFs4LDksIiIsMCx7InN0eWxlIjp7ImhlYWQiOnsibmFtZSI6Im5vbmUifX19XSxbOSw3LCIiLDAseyJzdHlsZSI6eyJoZWFkIjp7Im5hbWUiOiJub25lIn19fV0sWzcsMTAsIiIsMCx7InN0eWxlIjp7ImhlYWQiOnsibmFtZSI6Im5vbmUifX19XSxbMTAsOCwiIiwwLHsic3R5bGUiOnsiaGVhZCI6eyJuYW1lIjoibm9uZSJ9fX1dLFsxMCw5LCIiLDEseyJzdHlsZSI6eyJoZWFkIjp7Im5hbWUiOiJub25lIn19fV0sWzEwLDExLCIiLDEseyJzdHlsZSI6eyJoZWFkIjp7Im5hbWUiOiJub25lIn19fV0sWzExLDgsIiIsMSx7InN0eWxlIjp7ImhlYWQiOnsibmFtZSI6Im5vbmUifX19XSxbMTEsOSwiIiwxLHsic3R5bGUiOnsiaGVhZCI6eyJuYW1lIjoibm9uZSJ9fX1dLFsxMiwxMywiIiwxLHsic3R5bGUiOnsiaGVhZCI6eyJuYW1lIjoibm9uZSJ9fX1dLFsxMywxNCwiIiwxLHsic3R5bGUiOnsiaGVhZCI6eyJuYW1lIjoibm9uZSJ9fX1dLFsxNCwxMiwiIiwxLHsic3R5bGUiOnsiaGVhZCI6eyJuYW1lIjoibm9uZSJ9fX1dLFsxMiwxNSwiIiwxLHsic3R5bGUiOnsiaGVhZCI6eyJuYW1lIjoibm9uZSJ9fX1dLFsxNSwxMywiIiwxLHsic3R5bGUiOnsiaGVhZCI6eyJuYW1lIjoibm9uZSJ9fX1dLFsxNSwxNCwiIiwxLHsic3R5bGUiOnsiaGVhZCI6eyJuYW1lIjoibm9uZSJ9fX1dLFsxNSwxNiwiIiwxLHsic3R5bGUiOnsiaGVhZCI6eyJuYW1lIjoibm9uZSJ9fX1dLFsxNiwxMywiIiwxLHsic3R5bGUiOnsiaGVhZCI6eyJuYW1lIjoibm9uZSJ9fX1dLFsxNiwxNCwiIiwxLHsic3R5bGUiOnsiaGVhZCI6eyJuYW1lIjoibm9uZSJ9fX1dLFsxNywxNSwiIiwxLHsic3R5bGUiOnsiaGVhZCI6eyJuYW1lIjoibm9uZSJ9fX1dLFsxNywxNiwiIiwxLHsic3R5bGUiOnsiaGVhZCI6eyJuYW1lIjoibm9uZSJ9fX1dLFsxNywxNCwiIiwxLHsic3R5bGUiOnsiaGVhZCI6eyJuYW1lIjoibm9uZSJ9fX1dXQ==
	& {v_2} &&& {v_2} \\
	&&&& {v_4} \\
	{v_3} && {v_1} & {v_3} && {v_1} \\
	& {v_2} &&&& {v_2} \\
	& {v_5} &&&& {v_5} \\
	&& {v_4} &&& {v_6} & {v_4} \\
	{v_3} &&& {v_1} & {v_3} &&& {v_1}
	\arrow[no head, from=1-2, to=3-1]
	\arrow[no head, from=3-1, to=3-3]
	\arrow[no head, from=3-3, to=1-2]
	\arrow[no head, from=3-6, to=1-5]
	\arrow[no head, from=1-5, to=3-4]
	\arrow[no head, from=3-4, to=3-6]
	\arrow[no head, from=3-6, to=2-5]
	\arrow[no head, from=1-5, to=2-5]
	\arrow[no head, from=2-5, to=3-4]
	\arrow[no head, from=7-4, to=4-2]
	\arrow[no head, from=4-2, to=7-1]
	\arrow[no head, from=7-1, to=7-4]
	\arrow[no head, from=7-4, to=6-3]
	\arrow[no head, from=6-3, to=4-2]
	\arrow[no head, from=6-3, to=7-1]
	\arrow[no head, from=6-3, to=5-2]
	\arrow[no head, from=5-2, to=4-2]
	\arrow[no head, from=5-2, to=7-1]
	\arrow[no head, from=7-8, to=4-6]
	\arrow[no head, from=4-6, to=7-5]
	\arrow[no head, from=7-5, to=7-8]
	\arrow[no head, from=7-8, to=6-7]
	\arrow[no head, from=6-7, to=4-6]
	\arrow[no head, from=6-7, to=7-5]
	\arrow[no head, from=6-7, to=5-6]
	\arrow[no head, from=5-6, to=4-6]
	\arrow[no head, from=5-6, to=7-5]
	\arrow[no head, from=6-6, to=6-7]
	\arrow[no head, from=6-6, to=5-6]
	\arrow[no head, from=6-6, to=7-5]
	\end{tikzcd}\]

	We can place the \(v_{n+1}\) vertex in the region bounded by \(v_{n}, v_{n - 1}, v_{n - 2}\). As repersented in the picture above.

	Since the vertices are connected to the \(3\) vertices to the left (\(v_{i - 3}, v_{i - 2}, v_{i - 1}\)) and right (\(v_{i + 1}, v_{i + 2}, v_{i + 3}\)). All vertices, except for the first \(3\) and last  \(3\), have degree \(6\). However, \(v_{i - 3}, v_{i - 2}, v_{i - 1}\) do not exist for \(i = 1\), \(v_{i - 3}, v_{i - 2}\) do not exist for \(i = 2\),  \(v_{i -3}\) does not exist for \(i = 3\), and analogously for \(i = n, n - 1, n - 2\); therefore, these \(6\) vertices have in total \(12\) (\(2(1 + 2 + 3)\)) less connections than if the aforementioned vertices existed. Therefore, the sum of the degrees of the vertices is \(6n - 12\) and by the degree formula \(2|E| = \sum_{v \in V} \deg(v)\), there must be \(|E| = 3n - 6\), and so \(G\) is maximal planar.
\end{proof}


\pagebreak

{\it Submit a final project proposal. See other document for instructions. This should be submitted separately from the HW assignment: one submission per group.}


\end{document}
