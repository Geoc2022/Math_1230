\documentclass[11pt]{article}

\author{Math 123}
\date{Due April 7, 2023 by midnight} 
\title{Final Project Proposal}

\usepackage{graphicx,xypic}
\usepackage{amsthm}
\usepackage{amsmath,amssymb}
\usepackage{amsfonts}
\usepackage{xcolor}
\usepackage[margin=1in]{geometry}
\usepackage[shortlabels]{enumitem}
\newtheorem{problem}{Problem}
\renewcommand*{\proofname}{{\color{blue}Solution}}


\usepackage{fancyhdr}
\pagestyle{fancy}
\rhead{Math 123, Final project proposal}

\setlength{\parindent}{0pt}
\setlength{\parskip}{1.25ex}

% tikz
\usepackage{tikz}
\usetikzlibrary{intersections, angles, quotes, positioning}
\usetikzlibrary{arrows.meta}
\usepackage{pgfplots}
\pgfplotsset{compat=1.13}


\tikzset{
	force/.style={thick, {Circle[length=2pt]}-stealth, shorten <=-1pt}
}

% quiver style
\usepackage{tikz-cd}
% `calc` is necessary to draw curved arrows.
\usetikzlibrary{calc}
% `pathmorphing` is necessary to draw squiggly arrows.
\usetikzlibrary{decorations.pathmorphing}

% A TikZ style for curved arrows of a fixed height, due to AndréC.
\tikzset{curve/.style={settings={#1},to path={(\tikztostart)
					.. controls ($(\tikztostart)!\pv{pos}!(\tikztotarget)!\pv{height}!270:(\tikztotarget)$)
					and ($(\tikztostart)!1-\pv{pos}!(\tikztotarget)!\pv{height}!270:(\tikztotarget)$)
					.. (\tikztotarget)\tikztonodes}},
	settings/.code={\tikzset{quiver/.cd,#1}
			\def\pv##1{\pgfkeysvalueof{/tikz/quiver/##1}}},
	quiver/.cd,pos/.initial=0.35,height/.initial=0}

% TikZ arrowhead/tail styles.
\tikzset{tail reversed/.code={\pgfsetarrowsstart{tikzcd to}}}
\tikzset{2tail/.code={\pgfsetarrowsstart{Implies[reversed]}}}
\tikzset{2tail reversed/.code={\pgfsetarrowsstart{Implies}}}
% TikZ arrow styles.
\tikzset{no body/.style={/tikz/dash pattern=on 0 off 1mm}}

\begin{document}

\maketitle

% You are required to put your name here:
{\bf\Large Name: Rafae Ash, George Chemmala} 


\vspace{.3in}
Instructions. Choose a final project and a final project partner (groups of 2 - anything different needs approval). I would like you to submit the following information:

\begin{enumerate}
	\item[(i)] \emph{Project title:} Knight's Tour
	\item[(ii)] \emph{One paragraph about the focus and scope of your project. (Don't be overly ambitious. The presentation is 10 minutes! You should pick a single question/problem/theorem to focus on.)}
		
		The presentation will focus on the Knight's tour problem and its relationship with graph theory. We will discuss how the problem of finding a knight's tour on a chessboard can be translated into a graph theory problem of finding a Hamiltonian path on the board, a subset of the Hamiltonian path problem, which in itself is a subset of the NP-hard traveling salesman problem. We will highlight the fact that while the traveling salesman problem is NP-hard, the Knight's tour problem is a special case that can be solved in linear time.

		Some topics we may briefly cover given time:
		\begin{enumerate}
			\item History
			\item Number of tours
		\end{enumerate}

	\item[(iii)] \emph{A list of at least two references you will use (not Wikipedia)}
		\begin{enumerate}
			\item[1.] \url{https://web.archive.org/web/20151010071713/http://faculty.olin.edu:80/~sadams/DM/ktpaper.pdf} (general overview)
			
			\item[2.] \url{http://www.mayhematics.com/t/t.htm} (examples/history)
			
			\item[3.] \url{https://www.tandfonline.com/doi/pdf/10.1080/0025570X.1991.11977627} (existence theorem)
			
			\item[4.] \url{https://www.sciencedirect.com/science/article/pii/0166218X9200170Q?via\%3Dihub}
		\end{enumerate}
\end{enumerate}

\end{document}
