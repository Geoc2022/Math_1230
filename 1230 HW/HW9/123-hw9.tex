\documentclass[11pt]{article}

\author{Math 123}
\date{Due April 14, 2023 by midnight} 
\title{Homework 9}

\usepackage{graphicx,xypic}
\usepackage{amsthm}
\usepackage{amsmath,amssymb}
\usepackage{amsfonts}
\usepackage{xcolor}
\usepackage[margin=1in]{geometry}
\usepackage[shortlabels]{enumitem}
\newtheorem{problem}{Problem}
\renewcommand*{\proofname}{{\color{blue}Solution}}


\usepackage{fancyhdr}
\pagestyle{fancy}
\rhead{Math 123, Homework 9}

\setlength{\parindent}{0pt}
\setlength{\parskip}{1.25ex}

% tikz
\usepackage{tikz}
\usetikzlibrary{intersections, angles, quotes, positioning}
\usetikzlibrary{arrows.meta}
\usepackage{pgfplots}
\pgfplotsset{compat=1.13}


\tikzset{
	force/.style={thick, {Circle[length=2pt]}-stealth, shorten <=-1pt}
}

% quiver style
\usepackage{tikz-cd}
% `calc` is necessary to draw curved arrows.
\usetikzlibrary{calc}
% `pathmorphing` is necessary to draw squiggly arrows.
\usetikzlibrary{decorations.pathmorphing}

% A TikZ style for curved arrows of a fixed height, due to AndréC.
\tikzset{curve/.style={settings={#1},to path={(\tikztostart)
					.. controls ($(\tikztostart)!\pv{pos}!(\tikztotarget)!\pv{height}!270:(\tikztotarget)$)
					and ($(\tikztostart)!1-\pv{pos}!(\tikztotarget)!\pv{height}!270:(\tikztotarget)$)
					.. (\tikztotarget)\tikztonodes}},
	settings/.code={\tikzset{quiver/.cd,#1}
			\def\pv##1{\pgfkeysvalueof{/tikz/quiver/##1}}},
	quiver/.cd,pos/.initial=0.35,height/.initial=0}

% TikZ arrowhead/tail styles.
\tikzset{tail reversed/.code={\pgfsetarrowsstart{tikzcd to}}}
\tikzset{2tail/.code={\pgfsetarrowsstart{Implies[reversed]}}}
\tikzset{2tail reversed/.code={\pgfsetarrowsstart{Implies}}}
% TikZ arrow styles.
\tikzset{no body/.style={/tikz/dash pattern=on 0 off 1mm}}

\begin{document}

\maketitle

% You are required to put your name here:
{\bf\Large Name: George Chemmala} 


\vspace{.3in}
Topics covered: Ramsey theory

Instructions: 
\begin{itemize}
\item This assignment must be submitted on Gradescope by the due date. 
\item If you collaborate with other students (which is encouraged!), please mention this somewhere on the assignment. 
\item If you are stuck, please ask for help (from me, a TA, a classmate). Use Campus wire!  
\item You may freely use any fact proved in class. In general, you should provide proof for facts used that were not proved in class. 
\item Please restrict your solution to each problem to a single page. Usually solutions can be even shorter than that. If your solution is very long, you should think more about how to express it concisely.
\end{itemize}


\pagebreak 


\begin{problem}
Prove that $R(k,\ell)\le{k+\ell-2\choose k-1}$ and deduce that $R(k,k)\le 4^k$. \footnote{Use induction for the first part. You will probably want to use some well-known facts about binomial coefficients, which you can google.} 
\end{problem}

\begin{proof} 
First consider \(R(n, 2) = n\). Suppose \(K_n\) does not have all edges of the same color (blue), then the must be an edge of a different color (red) and so there exists \(K_2\) with red edges. However, in \(K_{n-1}\) we could make all the edges the same color (blue) and it wouldn't meet either criteria.

We prove $R(k,\ell)\le{k+\ell-2\choose k-1}$ by induction (without loss of generality let \(k \geq \ell\)):
\begin{enumerate}[align=left]
	\item[\emph{Base Case:}] (\(\ell = 2\))

	Earlier we showed \(R(k, 2) = k\) then \(R(k, 2) = k \leq {k+2-2\choose k-1} = {k\choose k-1} = {k \choose 1} = k\) 
	\item[\emph{Inductive Step:}]  

	By Ramsey's theorem we know:
	\[
		R(k, l) \leq R(k - 1, l) + R(k, l - 1)
	\]

	Therefore, using the inductive hypothesis:
	\[
		R(k, l) \leq R(k - 1, l) + R(k, l - 1) \leq {k+\ell-3\choose k-2} + {k+\ell-3\choose k-1} = {k+\ell-2\choose k-1} 
	\]
\end{enumerate}  

To find the \(4^k\) upper bound we can evaluate \((x+1)^{2k-2}\) at \(x = 1\) since one of the coeffecents of \((x+1)^{2k-2}\) must be \({k+k-2\choose k-1} = {2k-2\choose k-1}\). Therefore, the upper bound must be \(2^{2k-2} = 2^{2(k-1)} = 4^{k-1} \leq 4^k\)

\end{proof}

\pagebreak


\begin{problem}
Use the pigeonhole principle\footnote{The pigeonhole principle says that if you put $m$ pigeons into $n$ holes, then there is a hole with at least $\lceil m/n\rceil$ pigeons. We used this implicitly multiple times in the lecture 4/4 (it may be helpful to look back over the notes to see where it was used). } to prove that every set of $n$ integers $\{a_1,\ldots,a_n\}$ contains a nonempty subset whose sum is divisible by $n$. \footnote{Hint: consider the partial sums $S_i=a_1+\cdots+a_i$.}\footnote{Remark: The set $\{1,n+1,2n+1,\ldots,(n-1)n+1\}$ shows you cannot improve this result to a set of $n-1$ integers.} Give a collection of $n-1$ integers with no such subset. 
\end{problem}

\begin{proof}
Let the partial sums be $S_i=a_1+\cdots+a_i$.

If any \(S_i\) is divisible by \(n\) then we are done, but assume this is not the case. Therefore, by modding out by \(n\), \(S_i\) are in \(n-1\) equivalence classes. Since there are \(n\)  partial sums one of the classes must contain two sums \(S_i, S_j\), by pigeonhole principle. Therefore, we can take the difference of the sums (\(a_i + \cdots + a_j\) where \(k > i\)) and see that it mod \(n\) is zero.

\(\{2,2,2,2\}\) is a set where no subset is divisible by \(4 + 1 = 5\) 
\end{proof}

\pagebreak


\begin{problem}
You're given two concentric discs, each with $20$ radial sectors of equal size. For each disc, $10$ sectors are painted red and $10$ blue, in some (arbitrary) arrangement. Prove that the two discs can be aligned so that at least $10$ sectors on the inner disc match colors with the corresponding sector on the outer disc. \footnote{Hint: There is a very short solution using the pigeonhole principle.} \footnote{Hint: count the total number of inner/outer sector pairs with matching colors.}
\end{problem}

\begin{proof}
Each section can have a corresponding match with 10 other colors and there are 20 sections, so there are 200 possible corresponding matches. Since there are 20 ways to match the inner wheel with the outer wheel, one position of the inner wheel must have \(200 / 20 = 10\) corresponding matches, by pigeonhole (suppose there is one position which has less than 10 corresponding matches, then, since the total is 200, there must be one with more than 10 corresponding matches).
\end{proof}

\pagebreak

\begin{problem}
Fix $r\ge2$ and let $R(k,\ldots,k)$ (with $r$ copies of $k$) denote the minimal $n$ so that any $r$-coloring of the edges of $K_n$ contains at a monochromatic $K_k$ (i.e.\ we are generalizing the Ramsey numbers to more colors). Prove that every $r$ coloring of the numbers $1,\ldots,R(3,\ldots,3)$ (with $r$ copies of $3$) contains a monochromatic $x,y,z$ so that $x+y=z$. \footnote{Hint: set $n=R(3,\ldots,3)$. Given a coloring of $1,\ldots,n$, construct an edge 2-coloring of $K_n$ so that a monochromatic triangle in $K_n$ corresponds to monochromatic $x,y,z$ with $x+y=z$.} \footnote{Hint: labelling the vertices of $K_n$ by $v_1,\ldots,v_n$, choose the coloring of the edge $v_iv_j$ using the color of $|j-i|$ as a guide.}
\end{problem}

\begin{proof}
Let \(n = R(3,\ldots,3)\). Take \(K_n\), we will color the edges of the vertices \(v_1 \cdots v_n\) based on a function that considers the distance \(|i - j|\) between two vertices \(v_i\) and \(v_j\).

Because \(K_n\) has \(R(3,\ldots,3)\) vertices there will be a monochromatic \(K_3\), \(\{v_a, v_b, v_c\}\), in \(K_n\). Therefore, there exists \(x = b - a, y = c - b, z = c - a\), so \(x + y = z\)
\end{proof}


\pagebreak

\begin{problem}
A theorem similar to Nepali's theorem says that for any embedding of $K_6$ in $\mathbb R^3$, there exists a pair of triangles that form a Hope link.\footnote{Google this.} Verify this result for the following set of 6 points in $\mathbb R^3$. 
\[A=(0,3,2), \>\>\>B=(-2,-6,0), \>\>\>C=(6,3,2), \>\>\>D=(-6,-10,7), \>\>\>E=(-9,-6,9),\>\>\>F=(3,-1,9)\]
To help solve this, use this Geo-gebra notebook: 
\texttt{https://www.geogebra.org/m/ng6pspkh} \footnote{Click the circles on the left to make line segments appear/disappear.} Include a screenshot in your solution. 
\end{problem}
\begin{proof} Here we can see a Hopf link with \(A,D,F\) and \(B,C,D\) 

\includegraphics*[scale=.2]{Images/Screenshot 2023-04-14 at 1.37.03 PM.png}
	
\includegraphics*[scale=.2]{Images/Screenshot 2023-04-14 at 1.37.19 PM.png}
	
\includegraphics*[scale=.2]{Images/Screenshot 2023-04-14 at 1.37.32 PM.png}

\end{proof}

\pagebreak



{\it Submit a final project outline. See other document for instructions. }

\end{document}
