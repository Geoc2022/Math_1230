\documentclass[11pt]{article}

\author{Math 123}
\date{Due February 17, 2023 by 5pm} 
\title{Homework 3}

\usepackage{graphicx,xypic}
\usepackage{amsthm}
\usepackage{amsmath,amssymb}
\usepackage{amsfonts}
\usepackage{xcolor}
\usepackage[margin=1in]{geometry}
\usepackage[shortlabels]{enumitem}
\newtheorem{problem}{Problem}
\renewcommand*{\proofname}{{\color{blue}Solution}}


\usepackage{fancyhdr}
\pagestyle{fancy}
\rhead{Math 123, Homework 3}

\setlength{\parindent}{0pt}
\setlength{\parskip}{1.25ex}

\input{../tikz_import.tex}

\begin{document}

\maketitle

% You are required to put your name here:
{\bf\Large Name: George Chemmala} 

% \input{../tikz_import.tex}

\vspace{.3in}
Topics covered: trees, Pr\"ufer codes, spanning trees, counting graphs

Instructions: 
\begin{itemize}
\item This assignment must be submitted on Gradescope by the due date. 
\item If you collaborate with other students (which is encouraged!), please mention this near the corresponding problems. You must type your solutions alone. 
\item If you are stuck, please ask for help (from me, a TA, a classmate). Use Campuswire!  
\end{itemize}
\pagebreak 



\begin{problem}
Determine which trees have Pr\"ufer codes that 
\begin{enumerate}[(a)]
\item contain only one value;
\item contain exactly two values;
\item have distinct values. 
\end{enumerate} 
You should explain your answer, but you don't need to give careful proof. 
\end{problem}

\begin{proof}
\begin{enumerate}
    \item[(a)] Let \(x\) be the repeated value, then we know that if it's in the code, a vertex of degree 1 (which do not show up in the code) must be connected to \(x\). If it's repeated we should add another vertex of degree 1 to \(x\) and so on. 

    % https://q.uiver.app/?q=WzAsMTQsWzcsMSwieCJdLFs3LDAsIlxcYnVsbGV0Il0sWzgsMCwiXFxidWxsZXQiXSxbOCwxLCJcXGJ1bGxldCJdLFs4LDIsIlxcYnVsbGV0Il0sWzcsMiwiXFxidWxsZXQiXSxbNiwyLCJcXGJ1bGxldCJdLFs2LDEsIlxcYnVsbGV0Il0sWzYsMCwiXFxidWxsZXQiXSxbMCwxLCJ4Il0sWzEsMSwiXFxidWxsZXQiXSxbMywxLCJ4Il0sWzMsMCwiXFxidWxsZXQiXSxbNCwxLCJcXGJ1bGxldCJdLFswLDEsIiIsMCx7InN0eWxlIjp7ImhlYWQiOnsibmFtZSI6Im5vbmUifX19XSxbMCwyLCIiLDIseyJzdHlsZSI6eyJoZWFkIjp7Im5hbWUiOiJub25lIn19fV0sWzAsMywiIiwyLHsic3R5bGUiOnsiaGVhZCI6eyJuYW1lIjoibm9uZSJ9fX1dLFswLDQsIiIsMix7InN0eWxlIjp7ImhlYWQiOnsibmFtZSI6Im5vbmUifX19XSxbMCw1LCIiLDIseyJzdHlsZSI6eyJoZWFkIjp7Im5hbWUiOiJub25lIn19fV0sWzAsNiwiIiwyLHsic3R5bGUiOnsiaGVhZCI6eyJuYW1lIjoibm9uZSJ9fX1dLFswLDcsIiIsMix7InN0eWxlIjp7ImhlYWQiOnsibmFtZSI6Im5vbmUifX19XSxbMCw4LCIiLDIseyJzdHlsZSI6eyJoZWFkIjp7Im5hbWUiOiJub25lIn19fV0sWzksMTAsIiIsMCx7InN0eWxlIjp7ImhlYWQiOnsibmFtZSI6Im5vbmUifX19XSxbMTEsMTIsIiIsMCx7InN0eWxlIjp7ImhlYWQiOnsibmFtZSI6Im5vbmUifX19XSxbMTEsMTMsIiIsMix7InN0eWxlIjp7ImhlYWQiOnsibmFtZSI6Im5vbmUifX19XV0=
\[\begin{tikzcd}
	&&& \bullet &&& \bullet & \bullet & \bullet \\
	x & \bullet && x & \bullet && \bullet & x & \bullet \\
	&&&&&& \bullet & \bullet & \bullet
	\arrow[no head, from=2-8, to=1-8]
	\arrow[no head, from=2-8, to=1-9]
	\arrow[no head, from=2-8, to=2-9]
	\arrow[no head, from=2-8, to=3-9]
	\arrow[no head, from=2-8, to=3-8]
	\arrow[no head, from=2-8, to=3-7]
	\arrow[no head, from=2-8, to=2-7]
	\arrow[no head, from=2-8, to=1-7]
	\arrow[no head, from=2-1, to=2-2]
	\arrow[no head, from=2-4, to=1-4]
	\arrow[no head, from=2-4, to=2-5]
\end{tikzcd}\]

    \item[(b)] Let \(x\) and \(y\) be repeated values, then we know if they are in the code, vertices of degree 1 (which  not show up in the code) must be connected to \(x\) and \(y\). If they are repeated we should add another vertex of degree 1 to \(x\) / \(y\) and so on like the previous case.

    % https://q.uiver.app/?q=WzAsMjIsWzcsMSwieCJdLFs3LDAsIlxcYnVsbGV0Il0sWzgsMCwiXFxidWxsZXQiXSxbOCwxLCJcXGJ1bGxldCJdLFs4LDIsIlxcYnVsbGV0Il0sWzcsMiwieSJdLFs2LDIsIlxcYnVsbGV0Il0sWzYsMSwiXFxidWxsZXQiXSxbNiwwLCJcXGJ1bGxldCJdLFswLDEsIngiXSxbMSwxLCJcXGJ1bGxldCJdLFszLDEsIngiXSxbMywwLCJcXGJ1bGxldCJdLFs0LDEsIlxcYnVsbGV0Il0sWzAsMiwieSJdLFsxLDIsIlxcYnVsbGV0Il0sWzMsMywiXFxidWxsZXQiXSxbMywyLCJ5Il0sWzQsMiwiXFxidWxsZXQiXSxbNywzLCJcXGJ1bGxldCJdLFs4LDMsIlxcYnVsbGV0Il0sWzYsMywiXFxidWxsZXQiXSxbMCwxLCIiLDAseyJzdHlsZSI6eyJoZWFkIjp7Im5hbWUiOiJub25lIn19fV0sWzAsMiwiIiwyLHsic3R5bGUiOnsiaGVhZCI6eyJuYW1lIjoibm9uZSJ9fX1dLFswLDMsIiIsMix7InN0eWxlIjp7ImhlYWQiOnsibmFtZSI6Im5vbmUifX19XSxbMCw1LCIiLDIseyJzdHlsZSI6eyJoZWFkIjp7Im5hbWUiOiJub25lIn19fV0sWzAsNiwiIiwyLHsic3R5bGUiOnsiaGVhZCI6eyJuYW1lIjoibm9uZSJ9fX1dLFswLDcsIiIsMix7InN0eWxlIjp7ImhlYWQiOnsibmFtZSI6Im5vbmUifX19XSxbMCw4LCIiLDIseyJzdHlsZSI6eyJoZWFkIjp7Im5hbWUiOiJub25lIn19fV0sWzksMTAsIiIsMCx7InN0eWxlIjp7ImhlYWQiOnsibmFtZSI6Im5vbmUifX19XSxbMTEsMTIsIiIsMCx7InN0eWxlIjp7ImhlYWQiOnsibmFtZSI6Im5vbmUifX19XSxbMTEsMTMsIiIsMix7InN0eWxlIjp7ImhlYWQiOnsibmFtZSI6Im5vbmUifX19XSxbMTQsMTUsIiIsMCx7InN0eWxlIjp7ImhlYWQiOnsibmFtZSI6Im5vbmUifX19XSxbMTYsMTcsIiIsMCx7InN0eWxlIjp7ImhlYWQiOnsibmFtZSI6Im5vbmUifX19XSxbMTcsMTgsIiIsMCx7InN0eWxlIjp7ImhlYWQiOnsibmFtZSI6Im5vbmUifX19XSxbMTEsMTcsIiIsMSx7InN0eWxlIjp7ImhlYWQiOnsibmFtZSI6Im5vbmUifX19XSxbNSw0LCIiLDIseyJzdHlsZSI6eyJoZWFkIjp7Im5hbWUiOiJub25lIn19fV0sWzUsMTksIiIsMix7InN0eWxlIjp7ImhlYWQiOnsibmFtZSI6Im5vbmUifX19XSxbNSwyMCwiIiwyLHsic3R5bGUiOnsiaGVhZCI6eyJuYW1lIjoibm9uZSJ9fX1dLFs1LDIxLCIiLDIseyJzdHlsZSI6eyJoZWFkIjp7Im5hbWUiOiJub25lIn19fV1d
\[\begin{tikzcd}
	&&& \bullet &&& \bullet & \bullet & \bullet \\
	x & \bullet && x & \bullet && \bullet & x & \bullet \\
	y & \bullet && y & \bullet && \bullet & y & \bullet \\
	&&& \bullet &&& \bullet & \bullet & \bullet
	\arrow[no head, from=2-8, to=1-8]
	\arrow[no head, from=2-8, to=1-9]
	\arrow[no head, from=2-8, to=2-9]
	\arrow[no head, from=2-8, to=3-8]
	\arrow[no head, from=2-8, to=3-7]
	\arrow[no head, from=2-8, to=2-7]
	\arrow[no head, from=2-8, to=1-7]
	\arrow[no head, from=2-1, to=2-2]
	\arrow[no head, from=2-4, to=1-4]
	\arrow[no head, from=2-4, to=2-5]
	\arrow[no head, from=3-1, to=3-2]
	\arrow[no head, from=4-4, to=3-4]
	\arrow[no head, from=3-4, to=3-5]
	\arrow[no head, from=2-4, to=3-4]
	\arrow[no head, from=3-8, to=3-9]
	\arrow[no head, from=3-8, to=4-8]
	\arrow[no head, from=3-8, to=4-9]
	\arrow[no head, from=3-8, to=4-7]
\end{tikzcd}\]

    \item[(c)] If each vertex appears once then every vertex expect for at most 2 have degree \(2\). Therefore, the code must result in a path. The two vertices which have degree less than two are the endpoints of the path and the rest of the vertices show up in the code with degree 2. 

    % https://q.uiver.app/?q=WzAsNSxbMCwwLCJ2XzEiXSxbMSwwLCJ2XzIiXSxbMiwwLCJcXGRvdHMiXSxbMywwLCJ2X3tuLTF9Il0sWzQsMCwidl9uIl0sWzAsMSwiIiwwLHsic3R5bGUiOnsiaGVhZCI6eyJuYW1lIjoibm9uZSJ9fX1dLFsxLDIsIiIsMCx7InN0eWxlIjp7ImhlYWQiOnsibmFtZSI6Im5vbmUifX19XSxbMiwzLCIiLDAseyJzdHlsZSI6eyJoZWFkIjp7Im5hbWUiOiJub25lIn19fV0sWzMsNCwiIiwwLHsic3R5bGUiOnsiaGVhZCI6eyJuYW1lIjoibm9uZSJ9fX1dXQ==
\[\begin{tikzcd}
	{v_1} & {v_2} & \dots & {v_{n-1}} & {v_n}
	\arrow[no head, from=1-1, to=1-2]
	\arrow[no head, from=1-2, to=1-3]
	\arrow[no head, from=1-3, to=1-4]
	\arrow[no head, from=1-4, to=1-5]
\end{tikzcd}\]

Here the code would be \((v_2, \ldots, v_{n-1})\) since these vertices all have degree 2
\end{enumerate}
\end{proof}

\newpage

\begin{problem}
Prove that if $T_1,\ldots,T_k$ are pairwise-intersecting subtrees of a tree $T$, then $T$ has a vertex that belongs to all of $T_1,\ldots,T_k$. \footnote{Remark: This is a graph-theoretic analog of Helly's theorem.} \footnote{Hint: use induction on $k$.}
\end{problem}

\begin{proof}
We will take a Helly's Theorem style approach:

\emph{Proof that 3 trees \(T_1, T_2, T_3\) which are subtrees of a tree \(T\) and have nonempty intersections with each other share a common vertex:}

By contradiction, suppose that there is no vertex that is shared among the 3 trees. Therefore, there are vertices \(v_1, v_2, v_3\) s.t. \(v_1 \in T_2 \cap T_3\), \(v_2 \in T_1 \cap T_3\), \(v_3 \in T_1 \cap T_2\), but they are not in all 3 trees.

Let \(p \in T_3\) be a path from \(v_1\) to \(v_2\). In this path there will be a sub path which will have one end point in \(T_1\) (from def of \(v_2\)) and another in \(T_2\) (from def of \(v_1\)), but the path is uniquely in \(T_3\).

\(v_3 \in T_1 \cap T_2\) is not in \(T_3\), so we can create an alternate path from the endpoints of \(p\) (which are also in either \(T_1\) or \(T_2\)) through \(v_3\); however this would mean that there exists a cycle using the path we just constructed and the path between \(x\) and \(y\) in \(T_3\). Therefore, this produces a contradiction since we assumed we were working in a tree but got a cycle.

Using this proof for 3 trees, we can apply it \(k\) trees using induction:

\emph{Base Case:} We have shown with 3 trees this is the case

\emph{Inductive Step:} Suppose this holds for \(k-1\) trees
then we can demonstrate the same proof as before with \(T_k\), \(T_n\), and \(T_m\) where \(T_n, T_m \in \{T_1 \ldots T_{k-1}\}\). Therefore, \(T_k\) shares a vertex with \(T_n\) and \(T_m\)

\end{proof}

\begin{problem}
Let $G_n$ be the graph whose vertices are orderings of the elements of $\{1,\ldots,n\}$ with $(a_1,\ldots,a_n)$ and $(b_1,\ldots,b_n)$ adjacent if they differ by switching a pair of adjacent entries.\footnote{Note: Here the sequence $(a_1,\ldots,a_n)$ does not have repetition. Each element of $\{1,\ldots,n\}$ appears exactly once.}
\begin{enumerate}[(a)]
\item The graph $G_3$ is isomorphic to a familiar graph. Which one is it? 
\item Show that $G_n$ is connected. \footnote{Tangential remark: here you are proving that a certain collection of permutations generate the symmetric group.} 
\end{enumerate} 
\end{problem}

\begin{proof}
\begin{enumerate}
    \item[(a)] By taking all the vertices in \(G_3\) and connecting the vertices that differ by an adjacent switch we see that it forms \(C_6\)
    
    % https://q.uiver.app/?q=WzAsNixbMSwwLCIxMjMiXSxbMiwwLCIyMTMiXSxbMywxLCIyMzEiXSxbMiwyLCIzMjEiXSxbMSwyLCIzMTIiXSxbMCwxLCIxMzIiXSxbMSwyLCIiLDAseyJzdHlsZSI6eyJoZWFkIjp7Im5hbWUiOiJub25lIn19fV0sWzIsMywiIiwwLHsic3R5bGUiOnsiaGVhZCI6eyJuYW1lIjoibm9uZSJ9fX1dLFszLDQsIiIsMCx7InN0eWxlIjp7ImhlYWQiOnsibmFtZSI6Im5vbmUifX19XSxbNCw1LCIiLDAseyJzdHlsZSI6eyJoZWFkIjp7Im5hbWUiOiJub25lIn19fV0sWzUsMCwiIiwwLHsic3R5bGUiOnsiaGVhZCI6eyJuYW1lIjoibm9uZSJ9fX1dLFswLDEsIiIsMCx7InN0eWxlIjp7ImhlYWQiOnsibmFtZSI6Im5vbmUifX19XV0=
\[\begin{tikzcd}
	& 123 & 213 \\
	132 &&& 231 \\
	& 312 & 321
	\arrow[no head, from=1-3, to=2-4]
	\arrow[no head, from=2-4, to=3-3]
	\arrow[no head, from=3-3, to=3-2]
	\arrow[no head, from=3-2, to=2-1]
	\arrow[no head, from=2-1, to=1-2]
	\arrow[no head, from=1-2, to=1-3]
\end{tikzcd}\]
    
    \item[(b)] To show that the graph is connected we can show that all the vertices have a path to the ordered vertex \((1,2\ldots, n)\). This is the case since we can use an algorithm like bubble sort. We can take any two adjacent elements and switch them if they are in reverse order. By repeating this process we will eventually place \(1\) at the beginning since it is smaller than all other elements, \(2\) at the 2nd position since it is smaller than all other elements expect \(1\), and so on until \(n\). Since each switch is a move to another vertex, the total switches are analogous to a path of switches to \((1,2\ldots, n)\). Since every vertex has a path to \((1,2\ldots, n)\) every vertex must have a path to another vertex through \((1,2\ldots, n)\). 
\end{enumerate}
\end{proof}


\begin{problem}
Use Cayley's formula to prove that the graph obtained from $K_n$ by deleting an edge has $(n-2)n^{n-3}$ spanning trees. 
\end{problem}

\begin{proof}
Cayley's formula gives us that there are \(n^{n-2}\) spanning trees in \(K_n\). Since each tree has \(n\) vertices, each tree has \(n-1\) edges. The odds of picking the removed edge from the edge set is \(1/{n \choose 2}\) so the odds picking a tree with the removed edge is \((n-1)/{n \choose 2} = \frac{2}{n}\). From here we see that the number of trees with the removed edge is \(n^{n-2} \frac{2}{n} = 2 n^{n-3}\). Therefore, the number of trees without the removed edge is \(n^{n-2} - 2 n^{n-3} = (n-2)n^{n-3}\)
\end{proof}

\begin{problem}
Call a graph "even" if every vertex has even degree. Prove that the number of even graphs with vertex set $\{1,\ldots,n\}$ is $2^{{n-1\choose 2}}$. \footnote{Hint: establish a bijection to the set of all graphs with vertex set $\{1,\ldots,n-1\}$. }
\end{problem}


\begin{proof}
Let \(X\) be the set of all graphs with vertex set \(V = {v_1, \ldots, v_{n-1}}\), and \(Y\) be the set of all even graphs with vertex set \(V = {v_1, \ldots, v_{n}}\). Notice that the size of \(X\) is \(2^{{n-1 \choose 2}}\).

From here we can observe a natural bijection between \(X\) and \(Y\) with adding/subtracting \(v_n\):
\begin{enumerate}
    \item[\(X \to Y\)] We can add a vertex \(v_n\) s.t. it connects to every odd degree vertex, of which there are an even number of them (proved in the last problem of the last pset). By adding a connection to an odd vertex to \(v_n\), the odd vertices become even and \(v_n\) is of even degree since it's connected to an even number of vertices. If two graphs in \(X\) are sent to the same graph in \(Y\) then they must be the same graph since \(v_n\) must be attached to the same vertices to result in the same graph (injective). 
    \item[\(Y \to X\)] We can delete the vertex \(v_n\) resulting in a graph in \(X\) by definition. If two graphs in \(Y\) are sent to the same graph in \(X\) then they must be the same graph since \(v_n\) must have been attached to the same vertices to result in the same graph (injective).
\end{enumerate}
Therefore, this is a bijection of finite sets, so \(Y\) must have the same size as \(X\), \(2^{{n-1 \choose 2}}\) 
\end{proof}


\begin{problem}
Consider an alternative version of Bridg-it, where the player that forms a path connecting their ends loses. Give a strategy that shows that Player $2$ can always win. \footnote{You may use the following fact, which is similar to one we proved: Given spanning trees, $T,T'\subset G$ and an edge $e$ of $T$ but not $T'$, there exists an edge of $e'$ of $T'$ but not $T$ such that $T'-e'+e$ is a spanning tree. (Note the difference between this and the statement we proved in class!)}
\end{problem} 

\begin{proof}
In class, we showed that, in the regular game, Player 1 has a dominant strategy by reconnecting a spanning tree using an edge from the other spanning tree.

We can use the same graph theory setting to work with the problem in the altered case, only this time when a player maintains their spanning tree they lose. To show the second player has a winning strategy in the altered game, we want to show that the second player can force the first player to maintain their spanning tree. Therefore, the second player should aim to remove edges that the first player could use to create a cycle, thereby breaking the spanning tree. This would guarantee that the two spanning trees \(T, T^\prime \subset G\) remain spanning trees. Moreover, with the second player playing this defensive position, the first player will have to eventually make edges that rebuild the spanning trees because given spanning trees, $T,T'\subset G$ and an edge $e$ of $T$ but not $T'$, there exists an edge of $e'$ of $T'$ but not $T$ such that $T'-e'+e$ is a spanning tree. 
\end{proof}

\end{document}
