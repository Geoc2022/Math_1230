\documentclass[11pt]{article}

\author{Math 123}
\date{Due April 21, 2023 by midnight} 
\title{Homework 10}

\usepackage{graphicx,xypic}
\usepackage{amsthm}
\usepackage{amsmath,amssymb}
\usepackage{amsfonts}
\usepackage{xcolor}
\usepackage[margin=1in]{geometry}
\usepackage[shortlabels]{enumitem}
\newtheorem{problem}{Problem}
\renewcommand*{\proofname}{{\color{blue}Solution}}


\usepackage{fancyhdr}
\pagestyle{fancy}
\rhead{Math 123, Homework 10}

\setlength{\parindent}{0pt}
\setlength{\parskip}{1.25ex}

\input{../tikz_import.tex}

\begin{document}

\maketitle

% You are required to put your name here:
{\bf\Large Name: George Chemmala} 


\vspace{.3in}
Topics covered: Ramsey theory, random graphs

Instructions: 
\begin{itemize}
\item This assignment must be submitted on Gradescope by the due date. 
\item If you collaborate with other students (which is encouraged!), please mention this near the corresponding problems. 
\item Some problems from this assignment come from West's book, as indicated next to the problem. In some cases, the statements on this assignment differ slightly from the book. 
\item If you are stuck please ask for help (from me or your classmates). Occasionally problems may require ingredients not discussed in the course. 
\item You may freely use any fact proved in class. In general, you should provide proof for facts used that were not proved in class. 
\end{itemize}

\pagebreak 

\begin{problem}
Prove $R(4,4)>17$ using the $17$-vertex graph described in class.\footnote{17 vertices around a circle; connected a given vertex to the vertices distance 1, 2, 4, 8 away.}
\end{problem}

\begin{proof}
Let all the vertices be labeled from \(v_0\) to \(v_16\). Now any vertex \(v_i\) is connected to \(v_{i \pm 1}, v_{i \pm 2}, v_{i \pm 4}, v_{i \pm 8}\) where the labels are \(\mod 17\). 

By rotational symmetry we can consider a vertex \(v_i\). Any \(K_4\) will correspond to \(v_i\) and 3 vertices from \(\{v_{i \pm 1}, v_{i \pm 2}, v_{i \pm 4}, v_{i \pm 8}\}\). To simplify notation let \(v_i = 0, v_{i \pm 1} = 1, v_{i \pm 2} = 2 \cdots\) 

Assuming we start with \(0\), we can notice if we choose \(1\) we are forced to choose at least one from \(\{4, 8\}\) but we cannot connect \(1\) to \(\{4, 8\}\) since they are either distance \(3, 7 \not\in \{1, 2, 4, 8\}\) away.  

Therefore, we are forced to choose \(2\) and connect to \(4, 8\). However, \(2\) is a distance \(6 \not\in \{1, 2, 4, 8\}\) from \(8\).

Therefore, there exist no combination of 3 vertices that connect, so there exist no \(K_4\) in the graph.


Now we must check the complement, where vertices are a  distance 3, 5, 6, 7 away. We will use the same notation.

Assuming we start with \(0\) we can if we choose \(3\) we are forced to choose at least one from \(\{5, 7\}\) but we cannot connect \(3\) to \(\{5, 7\}\) since they are either distance \(2, 4 \not\in \{3,5,6,7\}\) away.  

Therefore, we are forced to choose \(5\) and connect to \(6, 7\). However, \(5\) is a distance \(1, 2 \not\in \{3,5,6,7\}\) from \(6, 7\).

Therefore, there exist no combination of 3 vertices that connect, so there exist no \(K_4\) in the graph.
\end{proof}

\pagebreak

\begin{problem}
Fix a graph $H$ with $k$ vertices. Prove that almost every graph contains $H$ as an induced subgraph.\footnote{Recall: given a collection of vertices in a graph $G$, the induced subgraph is the subgraph consisting of those vertices and all the edges between them that belong to $G$.} \footnote{Hint: Decompose the vertices into groups of size $k$. Consider the event that one these groups spans $H$.}
\end{problem}

\begin{proof}
I'm a little confused what the problem is asking but:

The limit of 
\[
	{n \choose k} (\frac{1}{2})^k
\]
approaches infinity as \(n\) tends to infinity since \({n \choose k}\) grows faster than \(2^{{k}}\) 
\end{proof}



\pagebreak

\begin{problem}
Recall that a graph $G$ satisfies property $(\star)$ if for any collection $u_1,\ldots,u_p$ and $v_1,\ldots,v_q$ of distinct vertices of $G$ there exists a vertex $z$ of $G$ so that $z$ is adjacent to all of the $u_i$ and to none of the $v_j$. Let $G_1,G_2$ be graphs whose vertex sets are countably infinite. Prove that if $G_1$ and $G_2$ satisfy $(\star)$, then $G_1$ and $G_2$ are isomorphic.\footnote{Hint: Enumerate the vertices of $G_1$ and $G_2$ by $x_1,x_2,\ldots$ and $y_1,y_2,\ldots$, respectively. Inductively define an isomorphism $f:V(G_1)\rightarrow V(G_2)$. On odd (resp.\ even) steps of the induction extend $f$ so that the smallest unmatched vertex of $V(G_1)$ (resp.\ $V(G_2)$) is in the domain (resp.\ image) of $f$.}
\end{problem}

\begin{proof}
Suppose $G_1$ and $G_2$ satisfy $(\star)$ and let the  vertices of $G_1$ and $G_2$ be enumerated by $x_1,x_2,\ldots$ and $y_1,y_2,\ldots$, respectively.

Let \(G_{1_i}, G_{2_i}\) be isomorphic subgraphs of \(G_1, G_2\). We will construct them by the following.

By induction:
\begin{enumerate}[align=left]
	\item[\emph{Base Case:}] (\(i = 1\))
	
	Here we have \(G_{1_1}, G_{2_1}\), both consisting of one vertex each. It is clear that these two vertices are isomorphic since they are not connected to anything
	\item[\emph{Inductive Step:}]  

	By the extension property, there exist \(x_{i} \in G_1, y_{i} \in G_2\) that are connected to the vertex sets of \(G_{1_i}, G_{2_i}\) but not to the remaining vertices in \(G_1, G_2\). Since these vertices \(x_i, y_i\) are connected to vertices which are isomorphic to each other, they themselves must be isomorphic, and we can add them to \(G_{1_i}, G_{2_i}\) to get \(G_{1_{i+1}}, G_{2_{i+1}}\)
\end{enumerate} 

\end{proof}

\pagebreak

\begin{problem}
Prove that the Radio graph has the following ``pigeonhole" property: For any partition of the vertex set $V=U_1\cup\cdots\cup U_m$, there exists $j$ so that the subgraph spanned by $U_j$ is isomorphic to $R$. 
\end{problem}

\begin{proof}
Suppose all partitions \(G_1, \cdots, G_m\) which correspond to vertex sets \(U_1,\cdots, U_m\) are not Rado.

Since \(G_1\) is not Rado, there exist subsets \(p, q\) of the vertex set \(U_1\) where no vertex in \(G_1\) is adjacent to \(p\) and not adjacent to \(q\).

Since \(G_2, \cdots G_m\) are not Rado, there exist subsets \(w, r\) of the vertex set \(U_2 \cup \cdots \cup U_m\) where no vertex in the corresponding graph \(G^\prime\) is adjacent to \(w\) and not adjacent to \(r\).

Now we have sets \(p \cup w\) and \(q \cup r\) s.t. they are disjoint, so in \(G\), which is Rado, there must be a vertex that connects to either \(p \cup w\) or \(q \cup r\). However, this vertex cannot be in either \(G_1\) or the rest of the graphs by the above reasoning, so we have a contradiction.
\end{proof}

\pagebreak


\begin{problem}[Bonus] Create a meme related to the course. Please submit to the campuswire page for everyone's enjoyment. 
\end{problem}
\begin{proof}
	https://campuswire.com/c/GCDD00E4D/feed/228
\end{proof}

\pagebreak
{\it Submit a draft of your final project slides. See other document for instructions.} 


\end{document}
